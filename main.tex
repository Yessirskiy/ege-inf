\documentclass[12pt]{article}

\usepackage{listings}
\usepackage{xcolor}

\usepackage{tabularx}
\newcolumntype{Y}{>{\centering\arraybackslash}X}
\newcounter{magicrownumbers}
\usepackage{multirow}
\newcommand\rownumber{\stepcounter{magicrownumbers}\arabic{magicrownumbers}}
\usepackage{xcolor,colortbl}

\definecolor{codegreen}{rgb}{0,0.6,0}
\definecolor{codegray}{rgb}{0.5,0.5,0.5}
\definecolor{codepurple}{rgb}{0.58,0,0.82}
\definecolor{backcolour}{rgb}{0.95,0.95,0.92}
\lstdefinestyle{Python}{
    language=Python,
    morekeywords={as, True, False},
    backgroundcolor=\color{backcolour},   
    commentstyle=\color{codegreen},
    keywordstyle=\color{magenta},
    numberstyle=\tiny\color{codegray},
    stringstyle=\color{codepurple},
    basicstyle=\ttfamily\footnotesize,
    breakatwhitespace=false,         
    breaklines=true,                 
    captionpos=b,                    
    keepspaces=true,                 
    numbers=left,                    
    numbersep=5pt,                  
    showspaces=false,                
    showstringspaces=false,
    showtabs=false,                  
    tabsize=2
}
\usepackage{minted}
\setminted{fontsize=\ttfamily\footnotesize}

\usepackage[margin=1in]{geometry} 
\usepackage{amsmath,amsthm,amssymb,scrextend}
\usepackage{centernot}
\usepackage{fancyhdr}
\pagestyle{fancy}

\usepackage[T2A]{fontenc}
\usepackage[utf8]{inputenc}

\newcommand{\cont}{\subseteq}
\usepackage{tikz}
\usepackage{circuitikz}
\usetikzlibrary{positioning, arrows.meta}
\usetikzlibrary{graphs, shapes.geometric}
\usepackage{pgfplots}
\usepackage{amsmath}
\usepackage[mathscr]{euscript}
\let\euscr\mathscr \let\mathscr\relax% just so we can load this and rsfs
\usepackage[scr]{rsfso}
\usepackage{amsthm}
\usepackage{amssymb}
\usepackage{multicol}
\usepackage[colorlinks=true, pdfstartview=FitV, linkcolor=blue,
citecolor=blue, urlcolor=blue]{hyperref}
\usepackage[russian]{babel}
\def\figureautorefname{Рис.}
\def\tableautorefname{Таблица}
\def\equationautorefname{Формула}
\def\subsubsectionautorefname{Подраздел}
\def\subsectionautorefname{Подраздел}


\usepackage[normalem]{ulem}
\usepackage{soul}
\DeclareMathOperator{\arcsec}{arcsec}
\DeclareMathOperator{\arccot}{arccot}
\DeclareMathOperator{\arccsc}{arccsc}
\newcommand{\ddx}{\frac{d}{dx}}
\newcommand{\dfdx}{\frac{df}{dx}}
\newcommand{\ddxp}[1]{\frac{d}{dx}\left( #1 \right)}
\newcommand{\dydx}{\frac{dy}{dx}}
\let\ds\displaystyle
\newcommand{\intx}[1]{\int #1 \, dx}
\newcommand{\intt}[1]{\int #1 \, dt}
\newcommand{\defint}[3]{\int_{#1}^{#2} #3 \, dx}
\newcommand{\imp}{\Rightarrow}
\newcommand{\un}{\cup}
\newcommand{\inter}{\cap}
\newcommand{\ps}{\mathscr{P}}
\newcommand{\set}[1]{\left\{ #1 \right\}}

\newtheorem{definition}{Определение}[section]
\counterwithin{table}{section}
\counterwithin{equation}{section}
\counterwithin{figure}{section}

\newtheoremstyle{problem_style}
  {}   % Space above
  {\topsep}   % Space below
  {\normalfont}  % Body font
  {}          % Indent amount
  {\bfseries} % Theorem head font
  {}         % Punctuation after theorem head
  {\newline}  % Space after theorem head
  {}          % Theorem head spec (can be left empty, meaning 'normal')
\theoremstyle{problem_style}
\newtheorem{problem}{Задание}[subsection]
\newtheorem{solution}{Решение}[subsection]


\usepackage{comment}
% Define a custom command to toggle solution mode
\newcommand{\solutionmode}[1]{%
    \ifnum#1=0
        \excludecomment{solution}
    \else
        \includecomment{solution}
    \fi
}

\setlength{\headheight}{15pt}
\pgfplotsset{compat=1.18}
\setlength{\parindent}{0pt}

\begin{document}
 
% EVERYTHING ABOVE THIS LINE IS JUST PREABLE, NO NEED TO MESS WITH IT.__________________________________________________________________________________________
%
\selectlanguage{russian}
\rhead{Автор: Добрыднев Николай (\textit{@dobrydnevk})}
\lhead{Подготовка к ЕГЭ (теория)}
%\chead{}
%\rhead{19 апреля 2024 г.} %\today

\tableofcontents
\newpage

\section{Задание 1. Анализ информационных моделей}
\subsection{Спецификация}
\textbf{Задание проверяет:}
Умение представлять и считывать данные в разных типах информационных моделей (схемы, карты, таблицы, графики и формулы)\\
\textbf{Уровень сложности:}
Базовый\\
\textbf{Максимальный балл:}
1\\
\textbf{Примерное время:}
3 минуты (1,5 минуты)
\subsection{Теория}
\begin{definition}
Граф — множество точек (вершин, узлов), которые соединяются множеством линий (ребрами).
\end{definition}
Примером графа из реального мира является карта страны: города это вершины, дороги между ними — ребра. Графы могут быть ориентированными и неориентированными.
\begin{definition}
Ориентированный граф — граф, в котором у каждого ребра есть направление. Ребра неориентированного графа не имеют направлений. 
\end{definition}
\begin{figure}[h]
    \centering
    \begin{minipage}[b]{0.45\textwidth}
        \centering
        \tikz {
            \node [circle, draw] (a) at (0,2) {A};
            \node [circle, draw] (b) at (2,4) {B};
            \node [circle, draw] (c) at (4,2) {C};
            \node [circle, draw] (d) at (2,0) {D};
            \node [circle, draw] (e) at (4,4) {E};
            \node [circle, draw] (f) at (0,4) {F};
            \graph [edges={thick, -{Latex[length=3mm]}}] {
                (a) -> (b) -> (c) -> (e),
                (d) -> (c),
                (d) -> (a) -> (f),
                (f) -> (b) -> (e)
            };
        }
        \caption{Ориентированный граф}
        \label{fig:dir_graph}
    \end{minipage}
    \hfill
    \begin{minipage}[b]{0.45\textwidth}
        \centering
        \tikz {
            \node [circle, draw] (a) at (0,2) {A};
            \node [circle, draw] (b) at (2,4) {B};
            \node [circle, draw] (c) at (4,2) {C};
            \node [circle, draw] (d) at (2,0) {D};
            \node [circle, draw] (e) at (4,4) {E};
            \node [circle, draw] (f) at (0,4) {F};
            \graph [edges={thick}] {
                (a) -- (b) -- (c) -- (e),
                (d) -- (c),
                (d) -- (a) -- (f),
                (f) -- (b) -- (e)
            };
        }
        \caption{Неориентированный граф}
        \label{fig:undir_graph}
    \end{minipage}
\end{figure}
\begin{definition}
Матрица смежности – представление графа в виде квадратной матрицы, где каждая строка и столбец соответствуют вершинам графа. Значение в определенной ячейке может соответствовать расстоянию  или говорить о наличии/отсутствии ребра между вершинами.
\end{definition}
Например, матрица смежности для \autoref{fig:undir_graph} может выглядеть следующим образом:
\begin{table}[h]
    \centering
    \begin{tabular}{c|cccccc}
         & a & b & c & d & e & f \\ \hline
        a & - & 12 &  & 4 &  & 5 \\
        b & 12 & - & 2 &  & 9 & 7 \\
        c &  & 2 & - & 6 & 10 &  \\
        d & 4 &  & 6 & - &  &  \\
        e &  & 9 & 10 &  & - &  \\
        f & 5 & 7 &  &  &  & - \\
    \end{tabular}
    \caption{Матрица смежности для \autoref{fig:undir_graph}}
    \label{tab:adjancency_matrix}
\end{table}\\
Как пример, данная таблица показывает, что расстояние между пунктами A и D равняется 4, а между B и F — 7. Отсутствие значения в ячейке, говорит об отсутствие ребра между точками (напр. A и C).\\
Обратите внимание, что данная матрица смежности симметрична относительно диагонали, это, как правило, говорит, что граф — неориентированный.\\
Стоит держать в голове, что расстояния, данные в матрице смежности не должны быть пропорциональны длинам ребер в графе.
\begin{definition}
Степень вершины графа (в неориентированном) — количество ребер, выходящих из вершины. В ориентированном, выделяют степени входа и выхода, количества входящих и выходящих ребер, соответственно. Для вершины \textit{V} степень будем записывать как \textit{deg(V)}.
\end{definition}

В заданиях первого типа дается матрица смежности, которую необходимо сопоставить с данным графом. При этом, есть графы и матрицы смежности, которые могут сопоставлены единственным образом (\ref{nonambig_match}) и те, которые не сопоставляются единственным образом (\ref{ambig_match}). Однако в обоих случаях есть возможность прийти к ответу на вопрос задачи.
\subsection{Однозначное соотнесение таблицы и графа}\label{nonambig_match}
Посмотрим на задачу данного подтипа:
\begin{problem}
На рисунке справа схема дорог Н-ского района изображена в виде графа; в таблице слева содержатся сведения о протяжённости каждой из этих дорог (в километрах).
\begin{figure}[h]
    \centering
    \begin{minipage}[t!]{0.45\textwidth}
        \centering
        \begin{tabular}{|c|c|c|c|c|c|c|c|} \hline
             & П1 & П2 & П3 & П4 & П5 & П6 & П7 \\ \hline
            П1 & - & 11 & 5 &  & 12 &  & \\ \hline
            П2 & 11 & - & 8 & 15 & & 23 & \\ \hline
            П3 & 5 & 8 & - &  & 10 &  & 7\\ \hline
            П4 &  & 15 &  & - &  & 10 & \\ \hline
            П5 & 12 &  & 10 &  & - &  & 11 \\ \hline
            П6 &  & 23 &  & 10 &  & - & \\ \hline
            П7 &  &  & 7 &  & 11 &  & -\\ \hline
        \end{tabular}
    \end{minipage}
    \hfill
    \begin{minipage}[t!]{0.45\textwidth}
        \centering
        \tikz {
            \node [circle, draw] (а) at (0,1) {А};
            \node [circle, draw] (б) at (1,2) {Б};
            \node [circle, draw] (в) at (2,1) {В};
            \node [circle, draw] (г) at (4,1) {Г};
            \node [circle, draw] (д) at (3,0) {Д};
            \node [circle, draw] (е) at (5,2) {Е};
            \node [circle, draw] (к) at (6,1) {К};
            \graph [edges={thick}] {
                (а) -- (б) -- (в) -- (г) -- (к) -- (е) -- (г),
                (г) -- (д) -- (в) -- (а),
                (д) -- (к)
            };
        }
    \end{minipage}
\end{figure}\\
Так как таблицу и схему рисовали независимо друг от друга, то нумерация населённых пунктов в таблице никак не связана с буквенными обозначениями на графе. Определите, какова длина дороги из пункта \textbf{В} в пункт \textbf{Г}. В ответе запишите целое число — так, как оно указано в таблице. \\
(\href{https://inf-ege.sdamgia.ru/}{Источник: РешуЕГЭ)} %https://inf-ege.sdamgia.ru/problem?id=18809
\end{problem}
\break
\begin{solution}
Решение всех задач данного типа должно начинаться с подсчитывания степени каждой вершины:
\begin{figure}[h]
    \centering
    \tikz {
        \node [circle, draw] (а) at (0,1) {А$_2$};
        \node [circle, draw] (б) at (1,2) {Б$_2$};
        \node [circle, draw] (в) at (2,1) {В$_4$};
        \node [circle, draw] (г) at (4,1) {Г$_4$};
        \node [circle, draw] (д) at (3,0) {Д$_3$};
        \node [circle, draw] (е) at (5,2) {Е$_2$};
        \node [circle, draw] (к) at (6,1) {К$_3$};
        \graph [edges={thick}] {
            (а) -- (б) -- (в) -- (г) -- (к) -- (е) -- (г),
            (г) -- (д) -- (в) -- (а),
            (д) -- (к)
        };
    }
\end{figure}\\
Как правило, в задачах данного типа, в графе существуют вершины, степени которых отличаются, от большинства остальных. В данной задаче, можно увидеть, что вершины В и Г - единственные, чьи степени равняются 4. Значит, в матрице смежности два пункта, имеющие по четыре значения в строке, соответствуют этим пунктам.\\
Не трудно заметить, что пункты П2 и П3 имеют по 4 значения в строчках. Так как, требуется лишь определить расстояние от В до Г, нам необязательно устанавливать четкое соответствие между ними (не важно, ищем ли мы расстояние от П2 до П3 или от П3 до П2). Из таблицы мы видим, что расстояние между этими пунктами равно 8.
\textbf{Ответ:} 8
\end{solution}
Решение заданий данного типа строится на поиске взаимосвязей в графе и матрице смежности. Самой очевидной взаимосвязью являются вершины с уникальными степенями, их всегда легко увидеть. Следующим шагом постарайтесь найти взаимосвязи между конкретными вершинами. Представим, что дан следующий граф и матрица смежности к нему:
\begin{figure}[h]
    \centering
    \begin{minipage}[t!]{0.45\textwidth}
        \centering
        \begin{tabular}{|c|c|c|c|c|c|} \hline
             & П1 & П2 & П3 & П4 & П5\\ \hline
            П1 & - & 11 & 5 & & \\ \hline
            П2 & 11 & - & 8 & 15 & \\ \hline
            П3 & 5 & 8 & - & & 1\\ \hline
            П4 &  & 15 &  & - & 9\\ \hline
            П5 &  &  & 1 & 9 & -\\ \hline
        \end{tabular}
    \end{minipage}
    \hfill
    \begin{minipage}[t!]{0.45\textwidth}
        \centering
        \tikz {
        \node [circle, draw] (а) at (0,1) {А$_2$};
        \node [circle, draw] (б) at (1,2) {Б$_3$};
        \node [circle, draw] (в) at (2,1) {В$_2$};
        \node [circle, draw] (г) at (4,1) {Г$_3$};
        \node [circle, draw] (д) at (3,2) {Д$_2$};
        \graph [edges={thick}] {
            (а) -- (б) -- (г) -- (б) -- (д),
            (а) -- (в) -- (г) -- (д)
        };
    }
    \end{minipage}
\end{figure}\\
Несложно, догадаться, что вершины Б и Г соответствуют пунктам П2 и П3 (или наоборот), так как это единственные, чьи степени равны 3 (по три записи в таблице). Однако, чтобы восстановить соответствие, этого недостаточно. Посмотрев на остальные вершины (А, В и Д), можно заметить, что только одна из них соединена с обеими вершинами, степени которых равны 3, это вершина Д. Найдем в таблице пункт, который соединён только с П2 и П3 - это П1, следовательно, П1 = Д.\\
Этот пример иллюстрирует, каким образом мы можем находить соответствия. Решим еще пару задач.
\begin{problem}
На рисунке схема дорог Н-ского района изображена в виде графа, в таблице содержатся сведения о длине этих дорог в километрах:
\begin{figure}[h]
    \centering
    \begin{minipage}[t!]{0.45\textwidth}
        \centering
        \begin{tabular}{|c|c|c|c|c|c|c|c|} \hline
             & П1 & П2 & П3 & П4 & П5 & П6 & П7 \\ \hline
            П1 & - & & 10 &  & &  & \\ \hline
            П2 &  & - & 20 &  & &  & \\ \hline
            П3 & 10 & 20 & - & 8 &  &  & \\ \hline
            П4 & & & 8 & - & 15 & 12 & \\ \hline
            П5 & & & & 15 & - & & \\ \hline
            П6 & & & & 12 & & - & 18 \\ \hline
            П7 & & & & & & 18 & -\\ \hline
        \end{tabular}
    \end{minipage}
    \hfill
    \begin{minipage}[t!]{0.45\textwidth}
        \centering
        \tikz {
            \node [circle, draw] (а) at (1,2) {А};
            \node [circle, draw] (б) at (5,2) {Б};
            \node [circle, draw] (в) at (2,1) {В};
            \node [circle, draw] (г) at (4,1) {Г};
            \node [circle, draw] (д) at (0,0) {Д};
            \node [circle, draw] (е) at (5,0) {Е};
            \node [circle, draw] (к) at (6,0) {К};
            \graph [edges={thick}] {
                (а) -- (в) -- (г) -- (б),
                (д) -- (в) -- (г) -- (е) -- (к)
            };
        }
    \end{minipage}
\end{figure}\\
Так как таблицу и схему рисовали независимо друг от друга, то нумерация населённых пунктов в таблице никак не связана с буквенными обозначениями на графе. Определите длину дороги из пункта \textbf{Г} в пункт \textbf{Е}. ВНИМАНИЕ! Длины отрезков на схеме не отражают длины дорог. \\
(\href{https://inf-ege.sdamgia.ru/}{Источник: РешуЕГЭ)} %https://inf-ege.sdamgia.ru/problem?id=13506
\end{problem}
\begin{solution}
Решение всех задач данного типа должно начинаться с подсчитывания степени каждой вершины:
\begin{figure}[h]
    \centering
    \tikz {
            \node [circle, draw] (а) at (1,2) {А$_1$};
            \node [circle, draw] (б) at (5,2) {Б$_1$};
            \node [circle, draw] (в) at (2,1) {В$_3$};
            \node [circle, draw] (г) at (4,1) {Г$_3$};
            \node [circle, draw] (д) at (0,0) {Д$_1$};
            \node [circle, draw] (е) at (5,0) {Е$_2$};
            \node [circle, draw] (к) at (7,0) {К$_1$};
            \graph [edges={thick}] {
                (а) -- (в) -- (г) -- (б),
                (д) -- (в) -- (г) -- (е) -- (к)
            };
        }
\end{figure}\\
Очевидно, вершиной с уникальной степенью (2) является Е. В таблице ей соответствует П6.
Обратим внимание, что только вершины В и Г имеют степени 3, при этом, только одна из них соединена с Е. Найдем в таблице вершину, которая имеет три ребра и соединена с П6 (Е) - П4. Следовательно, вершине Г соответствует П4.\\
Обе вершины (Г и Е) установлены, из таблицы найдем расстояние между ними (П4 и П6) - 12.\\
\textbf{Ответ:} 12
\end{solution}
\break
\begin{problem}
На рисунке схема дорог Н-ского района изображена в виде графа, в таблице содержатся сведения о длине этих дорог в километрах:
\begin{figure}[h]
    \centering
    \begin{minipage}[t!]{0.45\textwidth}
        \centering
        \begin{tabular}{|c|c|c|c|c|c|c|c|}\hline
        & П1 & П2 & П3 & П4 & П5 & П6 & П7 \\ \hline
        П1 & - & 45 & - & 10 & - & - & - \\ \hline
        П2 & 45 & - & - & 40 & - & 55 & - \\ \hline
        П3 & - & - & - & - & 15 & 60 & - \\ \hline
        П4 & 10 & 40 & - & - & - & 20 & 35 \\ \hline
        П5 & - & - & 15 & - & - & 55 & - \\ \hline
        П6 & - & 55 & 60 & 20 & 55 & - & 45 \\ \hline
        П7 & - & - & - & 35 & - & 45 & - \\ \hline
    \end{tabular}
    \end{minipage}
    \hfill
    \begin{minipage}[t!]{0.45\textwidth}
        \centering
        \tikz {
            \node [circle, draw] (а) at (-1,1) {А};
            \node [circle, draw] (б) at (0,2) {Б};
            \node [circle, draw] (в) at (1,1) {В};
            \node [circle, draw] (г) at (2,0) {Г};
            \node [circle, draw] (д) at (2,2) {Д};
            \node [circle, draw] (е) at (3,1) {Е};
            \node [circle, draw] (к) at (5,1) {К};
            \graph [edges={thick}] {
                (а) -- (б) -- (в) -- (а),
                (в) -- (д) -- (е) -- (в),
                (в) -- (г) -- (е) -- (к) -- (г),
            };
        }
    \end{minipage}
\end{figure}\\
Так как таблицу и схему рисовали независимо друг от друга, то нумерация населённых пунктов в таблице никак не связана с буквенными обозначениями на графе. Определите, какова длина дороги из пункта \textbf{К} в пункт \textbf{Е}. В ответе запишите целое число – так, как оно указано в таблице.\\
(\href{https://inf-ege.sdamgia.ru/}{Источник: РешуЕГЭ)} %https://inf-ege.sdamgia.ru/problem?id=13506
\end{problem}
\begin{solution}
Решение всех задач данного типа должно начинаться с подсчитывания степени каждой вершины:
\begin{figure}[h]
    \centering
    \tikz {
        \node [circle, draw] (а) at (-1,1) {А$_2$};
        \node [circle, draw] (б) at (0,2) {Б$_2$};
        \node [circle, draw] (в) at (1,1) {В$_5$};
        \node [circle, draw] (г) at (2,0) {Г$_3$};
        \node [circle, draw] (д) at (2,2) {Д$_2$};
        \node [circle, draw] (е) at (3,1) {Е$_4$};
        \node [circle, draw] (к) at (5,1) {К$_2$};
        \graph [edges={thick}] {
            (а) -- (б) -- (в) -- (а),
            (в) -- (д) -- (е) -- (в),
            (в) -- (г) -- (е) -- (к) -- (г),
        };
    }
\end{figure}\\
Обе вершены Г и Е имеют уникальные степени 3 и 4, соответственно. В таблице это пункты П2 и П4. Так как вершина К только соединена с Е и Г, в матрице достаточно найти пункт, соединяющийся только с П2 и П4 - это П1.\\
Обе вершины (К и Е) установлены, из таблицы найдем расстояние между ними (П4 и П1) - 10.\\
\textbf{Ответ:} 10
\end{solution}
\break
\begin{problem}
На рисунке справа схема дорог Н-ского района изображена в виде графа, в таблице содержатся сведения о длинах этих дорог (в километрах). 
\begin{figure}[h]
    \centering
    \begin{minipage}[t!]{0.45\textwidth}
        \centering
        \begin{tabular}{|c|c|c|c|c|c|c|c|}\hline
        & П1 & П2 & П3 & П4 & П5 & П6 & П7 \\ \hline
        П1 & - & & & & & 12 & 7 \\ \hline
        П2 & & - & & & 10 & 11 & 9 \\ \hline
        П3 & & & - & 5 & 6 & 3 & \\ \hline
        П4 & & & 5 & - & 15 & & \\ \hline
        П5 & & 10 & 6 & 15 & - & & \\ \hline
        П6 & 12 & 11 & 3 & & & - & \\ \hline
        П7 & 7 & 9 & & & & & - \\ \hline
        \end{tabular}
    \end{minipage}
    \hfill
    \begin{minipage}[t!]{0.45\textwidth}
        \centering
        \tikz {
            \node [circle, draw] (а) at (-2,1) {А};
            \node [circle, draw] (б) at (0,2) {Б};
            \node [circle, draw] (в) at (0,0) {В};
            \node [circle, draw] (г) at (2,2) {Г};
            \node [circle, draw] (д) at (2,0) {Д};
            \node [circle, draw] (е) at (4,2) {Е};
            \node [circle, draw] (з) at (4,0) {З};
            \graph [edges={thick}] {
                (а) -- (б) -- (в) -- (а),
                (б) -- (г) -- (д) -- (в),
                (г) -- (е) -- (з) -- (д),
            };
        }
    \end{minipage}
\end{figure}\\
Так как таблицу и схему рисовали независимо друг от друга, то нумерация населённых пунктов в таблице никак не связана с буквенными обозначениями на графе. В таблице в левом столбце указаны номера пунктов, откуда совершается движение, в первой строке — куда. Найдите сумму длин дорог из пункта \textbf{Г} в пункт \textbf{Е} и из пункта \textbf{Д} в \textbf{З}:\\
(\href{https://inf-ege.sdamgia.ru/}{Источник: РешуЕГЭ)} %https://inf-ege.sdamgia.ru/problem?id=13506
\end{problem}
\begin{solution}
Решение всех задач данного типа должно начинаться с подсчитывания степени каждой вершины:
\begin{figure}[h]
    \centering
    \tikz {
        \node [circle, draw] (а) at (-2,1) {А$_2$};
        \node [circle, draw] (б) at (0,2) {Б$_3$};
        \node [circle, draw] (в) at (0,0) {В$_3$};
        \node [circle, draw] (г) at (2,2) {Г$_3$};
        \node [circle, draw] (д) at (2,0) {Д$_3$};
        \node [circle, draw] (е) at (4,2) {Е$_2$};
        \node [circle, draw] (з) at (4,0) {З$_2$};
        \graph [edges={thick}] {
            (а) -- (б) -- (в) -- (а),
            (б) -- (г) -- (д) -- (в),
            (г) -- (е) -- (з) -- (д),
        };
    }
\end{figure}\\
Вершины А, Е и З имеют уникальные степени (2) (в таблице: П1, П4, П7), однако заметим, что из всех трех, только вершина А не имеет общего ребра с вершиной, у которой степень тоже 2 (Е и З соединенны между собой). Соответственно, в таблице это - П4.\\
Тогда, пункты П1 и П7 соответствуют Е и З (без понимания, кто есть кто). Тогда по таблице нам известно, что есть соответствие между ребрами ГЕ, ЕЗ, ЗД и П7-П1, П7-П2, П1-П6).Сложим три значения из таблицы ($12 + 7 + 9 = 28)$ и получим суммарную длину трех ребер. Однако ребро ЕЗ нас не интересует, но его мы можем установить наверняка П1-П7. Тогда $28 - 7 = 21$\\
\textbf{Ответ:} 21
\end{solution}
\break
\subsection{Неоднозначное соотнесение таблицы и графа}\label{ambig_match}
В задачах этого типа, как правило, матрица смежности заполняется звёздочками на тех местах, где подразумевается ребро. Разберем пример:
\begin{problem}
На рисунке слева изображена схема дорог N-ского района. В таблице звёздочкой обозначено наличие дороги из одного населённого пункта в другой. Отсутствие звёздочки означает, что такой дороги нет.
\begin{figure}[h]
    \centering
    \begin{minipage}[t!]{0.45\textwidth}
        \centering
        \begin{tabular}{|c|c|c|c|c|c|c|c|}\hline
        & 1 & 2 & 3 & 4 & 5 & 6 & 7 \\ \hline
        1 & - & & * & & &  & * \\ \hline
        2 & & - &  & & * &  & * \\ \hline
        3 & * & & - & * & & & * \\ \hline
        4 & & & * & - & & & * \\ \hline
        5 & & * & & & - & * & \\ \hline
        6 & & & & & * & - & * \\ \hline
        7 & * & * & * & * & & * & - \\ \hline
        \end{tabular}
    \end{minipage}
    \hfill
    \begin{minipage}[t!]{0.45\textwidth}
        \centering
        \tikz {
            \node [circle, draw] (a) at (-2,1) {A};
            \node [circle, draw] (b) at (1,2) {B};
            \node [circle, draw] (c) at (-1,2) {C};
            \node [circle, draw] (d) at (0,1) {D};
            \node [circle, draw] (e) at (1,0) {E};
            \node [circle, draw] (f) at (2,1) {F};
            \node [circle, draw] (g) at (-1,0) {G};
            \graph [edges={thick}] {
                (a) -- (c) -- (d) -- (g) -- (a) -- (d),
                (d) -- (b) -- (f) -- (e) -- (d)
            };
        }
    \end{minipage}
\end{figure}\\
Каждому населённому пункту на схеме соответствует его номер в таблице, но неизвестно какой именно номер. Определите, какие номера населённых пунктов в таблице могут соответствовать населённым пунктам \textbf{B} и \textbf{E} на схеме. В ответе запишите эти два номера в возрастающем порядке без пробелов и знаков препинания.\\
(\href{https://inf-ege.sdamgia.ru/}{Источник: РешуЕГЭ)} %https://inf-ege.sdamgia.ru/problem?id=68502
\begin{solution}
Решение всех задач данного типа должно начинаться с подсчитывания степени каждой вершины:
\begin{figure}[h]
    \centering
    \tikz {
        \node [circle, draw] (a) at (-2,1) {A$_3$};
        \node [circle, draw] (b) at (1,2) {B$_2$};
        \node [circle, draw] (c) at (-1,2) {C$_2$};
        \node [circle, draw] (d) at (0,1) {D$_5$};
        \node [circle, draw] (e) at (1,0) {E$_2$};
        \node [circle, draw] (f) at (2,1) {F$_2$};
        \node [circle, draw] (g) at (-1,0) {G$_2$};
        \graph [edges={thick}] {
            (a) -- (c) -- (d) -- (g) -- (a) -- (d),
            (d) -- (b) -- (f) -- (e) -- (d)
        };
    }
\end{figure}\\
Вершина D - единственная, чья степень равна 5. Из таблицы, очевидно, что ей соответствует 7. Таким же образом, можно установить, что вершине A соответствует 3.\\
Обратим внимание, что интересующие нас вершины B и E соединенны ребром с G (7) и не соединенны с A (3), это свойство есть только у них. Найдем по таблице, для каких пунктов это выполняется: 2 и 6.\\
\textbf{Ответ:} 26
\end{solution}
\break
\end{problem}


\begin{problem}
На рисунке слева изображена схема дорог Н-ского района, в таблице звёздочкой обозначено наличие дороги из одного населённого пункта в другой. Отсутствие звёздочки означает, что такой дороги нет.
\begin{figure}[h]
    \centering
    \begin{minipage}[t!]{0.45\textwidth}
        \centering
        \begin{tabular}{|c|c|c|c|c|c|c|c|}\hline
        & 1 & 2 & 3 & 4 & 5 & 6 & 7 \\ \hline
        1 & - &  & * & * &  &  & * \\ \hline
        2 &  & - & * &  & * & * &  \\ \hline
        3 & * & * & - & * & * & * & * \\ \hline
        4 & * &  & * & - &  &  &  \\ \hline
        5 &  & * & * &  & - &  &  \\ \hline
        6 &  & * & * &  &  & - & * \\ \hline
        7 & * &  & * &  &  & * & - \\ \hline
        \end{tabular}
    \end{minipage}
    \hfill
    \begin{minipage}[t!]{0.45\textwidth}
        \centering
        \tikz {
            \node [circle, draw] (a) at (-1,0) {A};
            \node [circle, draw] (b) at (1,0) {B};
            \node [circle, draw] (c) at (1,2) {C};
            \node [circle, draw] (d) at (-1,2) {D};
            \node [circle, draw] (e) at (2,1) {E};
            \node [circle, draw] (f) at (0,1) {F};
            \node [circle, draw] (g) at (-2,1) {G};
            \graph [edges={thick}] {
                (f) -- (c) -- (d) -- (f) -- (d) -- (g) -- (f),
                (g) -- (a) -- (f) -- (e) -- (b) -- (f),
                (a) -- (b)
            };
        }
    \end{minipage}
\end{figure}\\
Каждому населённому пункту на схеме соответствует его номер в таблице, но неизвестно какой именно номер. Определите, какие номера населённых пунктов в таблице могут соответствовать населённым пунктам \textbf{A} и \textbf{G} на схеме. В ответе запишите эти два номера в возрастающем порядке без пробелов и знаков препинания.\\
(\href{https://inf-ege.sdamgia.ru/}{Источник: РешуЕГЭ)} %https://inf-ege.sdamgia.ru/problem?id=68502
\begin{solution}
Решение всех задач данного типа должно начинаться с подсчитывания степени каждой вершины:
\begin{figure}[h]
    \centering
    \tikz {
        \node [circle, draw] (a) at (-1,0) {A$_3$};
        \node [circle, draw] (b) at (1,0) {B$_3$};
        \node [circle, draw] (c) at (1,2) {C$_2$};
        \node [circle, draw] (d) at (-1,2) {D$_3$};
        \node [circle, draw] (e) at (2,1) {E$_2$};
        \node [circle, draw] (f) at (0,1) {F$_6$};
        \node [circle, draw] (g) at (-2,1) {G$_3$};
        \graph [edges={thick}] {
            (f) -- (c) -- (d) -- (f) -- (d) -- (g) -- (f),
            (g) -- (a) -- (f) -- (e) -- (b) -- (f),
            (a) -- (b)
        };
    }
\end{figure}\\
Вершина D - единственная, чья степень равна 6. Из таблицы, очевидно, что ей соответствует 3. Таким же образом, можно установить, что вершинам C и E соответствуют пункты 4 и 5 (или наоборот).\\
Обратим внимание, что интересующие нас вершины A и G соединенны ребром только с теми вершинами, степени, которых больше или равны 3, следовательно, интересующие нас пункты не имеют ребер с пунктами 4 и 5 - это пункты 6 и 7.\\
\textbf{Ответ:} 67
\end{solution}
\break
\end{problem}

\begin{problem}
На рисунке слева изображена схема дорог Н-ского района, в таблице звёздочкой обозначено наличие дороги из одного населённого пункта в другой. Отсутствие звёздочки означает, что такой дороги нет.
\begin{figure}[h]
    \centering
    \begin{minipage}[t!]{0.45\textwidth}
        \centering
        \begin{tabular}{|c|c|c|c|c|c|c|c|c|} \hline
        & П1 & П2 & П3 & П4 & П5 & П6 & П7 & П8 \\ \hline
        П1 & - &  &  & 15 & 29 & 31 &  &  \\ \hline
        П2 &  & - & 18 &  & 30 &  & 25 &  \\ \hline
        П3 &  & 18 & - &  &  &  & 33 & 24 \\ \hline
        П4 & 15 &  &  & - &  & 21 &  &  \\ \hline
        П5 & 29 & 30 &  &  & - & 14 &  & 27 \\ \hline
        П6 & 31 &  &  & 21 & 14 & - & 23 &  \\ \hline
        П7 &  & 25 & 33 &  &  & 23 & - & 12 \\ \hline
        П8 &  &  & 24 &  & 27 &  & 12 & - \\ \hline
        \end{tabular}
    \end{minipage}
    \hfill
    \begin{minipage}[t!]{0.45\textwidth}
        \centering
        \tikz {
            \node [circle, draw] (а) at (-3,2) {А};
            \node [circle, draw] (б) at (-1,2) {Б};
            \node [circle, draw] (в) at (1,2) {В};
            \node [circle, draw] (г) at (3,2) {Г};
            \node [circle, draw] (д) at (-3,0) {Д};
            \node [circle, draw] (е) at (-1,0) {Е};
            \node [circle, draw] (ж) at (1,0) {Ж};
            \node [circle, draw] (и) at (3,0) {И};
            \graph [edges={thick}] {
                (д) -- (б) -- (а) -- (д) -- (е),
                (е) -- (в) -- (б) -- (е) -- (ж),
                (ж) -- (в) -- (г) -- (и) -- (в),
                (ж) -- (и)
            };
        }
    \end{minipage}
\end{figure}\\
Так как таблицу и схему рисовали независимо друг от друга, нумерация населённых пунктов в таблице никак не связана с буквенными обозначениями на графе. Известно, что одна дорога в таблице отмечена неверно: из двух пунктов, которые соединяет эта дорога, правильно указан только один. В результате в одном из пунктов в таблице одной дороги не хватает, а в другом — появилась лишняя дорога. Определите длину дороги \textbf{ГИ}.\\
(\href{https://inf-ege.sdamgia.ru/}{Источник: РешуЕГЭ)} %https://inf-ege.sdamgia.ru/problem?id=68502
\begin{solution}
Решение всех задач данного типа должно начинаться с подсчитывания степени каждой вершины:
\begin{figure}[h]
    \centering
    \tikz {
        \node [circle, draw] (а) at (-3,2) {А$_2$};
        \node [circle, draw] (б) at (-1,2) {Б$_4$};
        \node [circle, draw] (в) at (1,2) {В$_5$};
        \node [circle, draw] (г) at (3,2) {Г$_2$};
        \node [circle, draw] (д) at (-3,0) {Д$_3$};
        \node [circle, draw] (е) at (-1,0) {Е$_4$};
        \node [circle, draw] (ж) at (1,0) {Ж$_3$};
        \node [circle, draw] (и) at (3,0) {И$_3$};
        \graph [edges={thick}] {
            (д) -- (б) -- (а) -- (д) -- (е),
            (е) -- (в) -- (б) -- (е) -- (ж),
            (ж) -- (в) -- (г) -- (и) -- (в),
            (ж) -- (и)
        };
    }
\end{figure}\\
В таблице 1 пункт, в котором 2 дороги,  4 пункта, в которых по 3 дороги, 3 пункта, в которых по 4 дороги и 0 пунктов, в которых 5 дорог.\\
Пункту П4 соответствует А, тогда П6 соответствует Б, а П1 соответствует Д. Б и Д оба ведут в П5, значит, П5 соответствует Е. П7 имеет 4 дороги, очевидно, что это и есть пункт с недостающей дорогой, а сам П7 соответствует В.\\
Теперь возможны 2 варианта. Либо П8 соответствует Ж и в П2 лишняя дорога, либо П2 соответствует Ж и в П8 лишняя дорога. Если П8 это Ж, то П3 соответствует И. Тогда П2 соответствует Г, и расстояние ГИ равно 18. Если П2 это Ж, то П3 соответствует И. Тогда П8 соответствует Г, и расстояние ГИ равно 24.\\
\textbf{Ответ:} 24 или 18
\end{solution}
\newpage
\end{problem}



%---------------------------------



\section{Задание 2. Построение таблиц истинности логических выражений}
\subsection{Спецификация}
\textbf{Задание проверяет:}
Умение строить таблицы истинности и логические схемы\\
\textbf{Уровень сложности:}
Базовый\\
\textbf{Максимальный балл:}
1\\
\textbf{Примерное время:}
3 минуты (1,5 минуты)

\subsection{Теория}
\begin{definition}
Булева алгебра – это раздел математической логики, в котором переменные могут иметь всего два значения: «истина» (обычно обозначается как 1) и «ложь» (обычно обозначается как 0). Булева алгебра также известна как булева логика, двоичная логика, двоичная алгебра, алгебра высказываний.
\end{definition}
Основными объектами булевой алгебры являются логические высказывания.
\begin{definition}
Логическое высказывание — это повествовательное предложение, в отношении которого можно однозначно сказать, истинно (1) оно или ложно (0).
\end{definition}
При этом, главным образом нас интересует значение выражение (истина или ложь), а не предмет высказывания. Пример высказывания: \textit{На улице идет дождь}. Это выражение истинно, если на улице действительно идет дождь и ложно, если это не так.\\
В какой-то момент, при решении простейших арифметических задач мы начали использовать \textit{переменные}, ведь гораздо удобнее записать просто \(x\) вместо \textit{данное кол-во яблок}, поступим таким же образом. 
\begin{definition}
Логические переменные – это переменные, которые могут принимать только два значения: истина (1) или ложь (0).
\end{definition}
Далее мы будем работать именно с логическими переменными, как правило, они обозначаются заглавными латинскими буквами, например: \(A\)  \\
Вспомним, что в начальной школе мы не остановились на том, что числа просто существуют, вместо этого мы продолжили изучать \textit{операции} над числами. Так нас научили их складывать, вычитать, перемножать и тд. Последуем этому примеру и изучим основные операции над высказываниями, их всего три:
\begin{itemize}
    \item Логическое отрицание (инверсия)
    \item Логическое сложение (дизъюнкция)
    \item Логическое умножение (конъюнкция)
\end{itemize}
\subsubsection{Логическое отрицание}
\begin{definition}
Логическое отрицание — унарная операция над высказыванием, результатом которой является высказывание, «противоположное» исходному по значению.
\end{definition}
Логическое сложение аналогично поведению обычного \textit{"не"} в повседневном языке: если значение высказывания было истинным (1), то после применения данной операции получится ложь (0), и наоборот. Например, для высказывания \textit{На улице идет дождь}, логическим отрицанием будет высказывание \textit{На улице НЕ идет дождь}.\\
Слово \textit{унарная} говорит о возможности применения этой операции лишь к одному высказыванию.  Эту операцию так же называют \textit{инверсией} или \textit{логическим НЕ}.\\
Есть несколько способов записи: \( \neg A \); \( \sim A \); \( \bar{A} \); \(not A\)

\subsubsection{Логическое сложение}
\begin{definition}
Логическое сложение — это бинарная логическая операция, результатом которой является истинное значение (1), если хотя бы одно из входных высказываний истинно, и ложное значение (0), если оба входных высказывания ложны.
\end{definition}
Логическое сложение аналогично поведению обычного \textit{"или"} в повседневном языке: оно утверждает, что если хотя бы одно из высказываний истинно, то итоговое утверждение также будет истинным. Например, если первое высказывание: \textit{Сейчас день}, а второе: \textit{Сейчас ночь}, то логическое сложение между ними будет истинным, так как хотя бы одно из высказываний истинно.\\
Эту операцию также называют \textit{дизъюнкцией} или \textit{логическим ИЛИ}. Она используется для объединения двух высказываний, проверяя хотя бы наличие одного из них.\\
Для высказываний $A, B$ может быть записана разными способами: \(A \lor B\); \( A + B \); \(A\,or\,B \)

\subsubsection{Логическое умножение}
\begin{definition}
Логическое умножение —  бинарная логическая операция, результатом которой является истинное значение (1) только в том случае, если оба входных высказывания истинны, в противном случае результат будет ложным (0).
\end{definition}
Логическое умножение аналогично поведению обычного \textit{"и"} в повседневном языке: оно утверждает, что оба высказывания должны быть истинны, чтобы итоговое утверждение также было истинным. Например, если первое высказывание: \textit{Мы живем в 21 веке}, а второе: \textit{Мы живем в первой половине века}, то логическое И между ними будет истинным, так как оба высказывания истинны.\\
Эту операцию также называют \textit{конъюнкцией} или \textit{логическим И}. Для высказываний $A, B$ может быть записана разными способами: \(A \land B\); \( A \cdot B \); \( A \& B \); \( A\,and\,B \)

\subsubsection{Логическая функция}
Как и арифметические операции, логические могут быть объеденины: \((A \land B) \lor C\)\\
Полученное выражение — новое высказывание, которое тоже принимает значение Истина или Ложь в зависимости от значений входных переменных. Для дальнейшего удобства, будем называть это новое высказывание — логической функцией (функция в алгебре — зависимость одной переменной величины, от другой/других переменных величин).
\begin{definition}
Логическая функция – это функция, которая принимает одно или несколько логических значений в качестве входных данных и возвращает логическое значение в качестве результата.
\end{definition} 
Пример записи логической функции: \(F(x, y, z) = x \land (y \lor \lnot z)\).
Так как при записи функции мы часто будем использовать несколько операций вместе, обозначим их приоритет:
\begin{table}[h]
    \centering
    \begin{tabular}{c|c}
         Инверсия & \(\lnot A\) \\ \hline
         Конъюнкция & \(A \land B\) \\ \hline
         Дизъюнкция & \(A \lor B\) \\ 
    \end{tabular}
    \caption{Приоритет основных логических операций}
    \label{tab:lops_priority_1}
\end{table}
\begin{definition}
Тавтология — логическая функция, которая принимает истинное значение при любой комбинации значений входных аргументов.
\end{definition}
\subsubsection{Таблица истинности}
Часто перебор всех значений функции в алгебре невозможен, так как их кол-во бесконечно. Однако значения логических функций часто перебираются без труда. Например, если функция принимает 3 логических значения на вход, то значений у данной функции 8, так как каждая входная переменная может принимать всего два значения, значит различных комбинаций входных значений \(2^3 = 8\).
В общем случае, логическая функция принимающая $n$ входных значений, имеет $2^n$ выходных значений.\\
Самым удобным способом представления значений функции в зависимости от значений аргументов является таблица истинности.
\begin{definition}
Таблица истинности — таблица содержащая все возможные комбинации входных переменных и соответствующее им значения функции на выходе.
\end{definition}
Составим таблицы истинности для трех основных операций:
\begin{figure}[h!]
    \centering
    \begin{minipage}{0.3\linewidth}
        \centering
        \begin{tabular}{|c|c|}
        \hline
        $A$ & $\neg A$ \\
        \hline
        1 & 0 \\
        0 & 1 \\
        \hline
        \end{tabular}
    \end{minipage}
    \hspace{0.5cm}
    \begin{minipage}{0.3\linewidth}
        \centering
        \begin{tabular}{|c|c|c|}
        \hline
        $A$ & $B$ & $A \lor B$ \\
        \hline
        0 & 0 & 0 \\
        0 & 1 & 1 \\
        1 & 0 & 1 \\
        1 & 1 & 1 \\
        \hline
        \end{tabular}
    \end{minipage}
    \hspace{0.5cm}
    \begin{minipage}{0.3\linewidth}
        \centering
        \begin{tabular}{|c|c|c|}
        \hline
        $A$ & $B$ & $A \land B$ \\
        \hline
        0 & 0 & 0 \\
        0 & 1 & 0 \\
        1 & 0 & 0 \\
        1 & 1 & 1 \\
        \hline
        \end{tabular}
    \end{minipage}
    \caption{Таблицы истинности для основных операций}
    \label{fig:truth_tables_1}
\end{figure}\\
В таблице 1 соответствует \textit{истине} и 0 соответствует \textit{лжи}. Например, в предпоследней строке в таблице с конъюнкцией записано: при \(A = 1\) и \(B = 0\), значении \(A \land B = 0\).\\
Таблицы истинности позволяют нам найти значения какой-либо функции и проанализировать ее:\\
Пусть дана функция \(F(x, y, z) = \lnot x \land y \land (x \lor z)\).
Построим для нее таблицу истинности следующим образом: сначала разобьём изначальную функции на части и заполним таблицу истинности для этих частей. После этого заполним таблицу для функции, по имеющимся значениям частей:
\begin{figure}[h!]
    \centering
        \begin{tabular}{|c|c|c|c|c|c|}
        \hline
        $x$ & $y$ & $z$ & \(\lnot x\) & \(x \lor z\) & \(F(x, y, z)\) \\
        \hline
        0 & 0 & 0 & 1 & 0 & 0\\
        0 & 0 & 1 & 1 & 1 & 0\\
        0 & 1 & 0 & 1 & 0 & 0\\
        0 & 1 & 1 & 1 & 1 & 1\\
        1 & 0 & 0 & 0 & 1 & 0\\
        1 & 0 & 1 & 0 & 1 & 0\\
        1 & 1 & 0 & 0 & 1 & 0\\
        1 & 1 & 1 & 0 & 1 & 0\\
        \hline
        \end{tabular}
\end{figure}
\subsubsection{Импликация}
В алгебре мы умеем возводить число в степень и считаем это арифметической операцией, хотя на самом деле это по сути, выполнение другой арифметической операции несколько раз (произведение). В математической логике так же существуют операции, которые были введены ради удобства, но все еще могут быть выражены через основные логические операции (инверсия, конъюнкция и дизъюнкция).
\begin{definition}
Импликация — это бинарная логическая операция, результатом которой является ложное значение (0) только в том случае, если первое высказывание истинно (1), а второе ложно (0); в остальных случаях результат истинный (1).
\end{definition}
Импликация часто интерпретируется как утверждение "если A, то B". Она утверждает, что если первое высказывание A истинно, то второе высказывание B также должно быть истинным, чтобы вся импликация была истинной. Например, если первое высказывание: \textit{Если идет дождь}, а второе: \textit{Тогда улица мокрая}, то импликация будет ложной только в том случае, если идет дождь, а улица не мокрая. В остальных случаях она будет истинной.\\
Эту операцию также называют \textit{условием} или \textit{следованием}. Она используется для установления логической зависимости между двумя высказываниями. Импликация записывается как \(A \Rightarrow B\).\\
Логическая функция \(\lnot A \lor B\) — запись импликации через основные логические операции, т.е. импликацию можно заменить этим выражением и функция не поменяет свои значения. Этим можно пользоваться при упрощении логического выражения.
\newpage
Составим таблицу истинности для импликации и ее записи через стандартные операции:
\begin{figure}[h!]
    \begin{minipage}{0.45\linewidth}
        \centering
        \begin{tabular}{|c|c|c|}
        \hline
        $A$ & $B$ & $A \Rightarrow B$ \\
        \hline
        0 & 0 & 1 \\
        0 & 1 & 1 \\
        1 & 0 & 0 \\
        1 & 1 & 1 \\
        \hline
        \end{tabular}
    \end{minipage}
    \hspace{0.5cm}
    \begin{minipage}{0.45\linewidth}
        \centering
        \begin{tabular}{|c|c|c|c|}
        \hline
        $A$ & $B$ & $\lnot A$ & $\lnot A \lor B$ \\
        \hline
        0 & 0 & 1 & 1\\
        0 & 1 & 1 & 1\\
        1 & 0 & 0 & 0\\
        1 & 1 & 0 & 1\\
        \hline
        \end{tabular}
    \end{minipage}
\end{figure}
\subsubsection{Эквиваленция}
\begin{definition}
Эквиваленция — это бинарная логическая операция, результатом которой является истинное значение (1) в случае, если оба высказывания имеют одинаковые значения, и ложное значение (0) в случае, если значения высказываний различны.
\end{definition}
Эквиваленция часто интерпретируется как утверждение "A тогда и только тогда, когда B". Она утверждает, что оба высказывания A и B должны быть либо истинными, либо ложными для того, чтобы вся эквиваленция была истинной. Например, если первое высказывание: \textit{Сегодня вторник}, а второе: \textit{Сегодня рабочий день}, то эквиваленция будет истинной, если оба высказывания истинны или оба ложны.\\
Эту операцию также называют \textit{равнозначностью} или \textit{логическим равенством}. Она используется для проверки эквивалентности двух высказываний. Эквиваленция записывается как \(A \Leftrightarrow B\); \(A \equiv B\) .\\
Логическая функция \((A \land B) \lor (A \land B)\) — запись эквиваленции через основные логические операции, т.е. эквиваленцию можно заменить этим выражением, и функция не поменяет свои значения. Этим можно пользоваться при упрощении логического выражения.
Составим таблицу истинности для эквиваленции и её записи через стандартные операции:
\begin{figure}[h!]
    \begin{minipage}{0.25\linewidth}
        \centering
        \begin{tabular}{|c|c|c|}
        \hline
        $A$ & $B$ & $A \Leftrightarrow B$ \\
        \hline
        0 & 0 & 1 \\
        0 & 1 & 0 \\
        1 & 0 & 0 \\
        1 & 1 & 1 \\
        \hline
        \end{tabular}
    \end{minipage}
    \hspace{0.2cm}
    \begin{minipage}{0.85\linewidth}
        \centering
        \begin{tabular}{|c|c|c|c|c|}
        \hline
        $A$ & $B$ & $A \land B$ & $\lnot A \land \lnot B$ & $(A \land B) \lor (\lnot A \land \lnot B)$ \\
        \hline
        0 & 0 & 0 & 1 & 1\\
        0 & 1 & 0 & 0 & 0\\
        1 & 0 & 0 & 0 & 0\\
        1 & 1 & 1 & 0 & 1\\
        \hline
        \end{tabular}
    \end{minipage}
\end{figure}

\subsubsection{Исключающее ИЛИ}
\begin{definition}
Исключающее ИЛИ — это бинарная логическая операция, результатом которой является истинное значение (1) в случае, если одно из входных высказываний истинно, а другое ложно. Если оба высказывания либо истинны, либо ложны, результат будет ложным (0).
\end{definition}
Исключающее ИЛИ часто интерпретируется как утверждение "либо A, либо B, но не оба". Оно утверждает, что одно из высказываний A или B должно быть истинным, а другое — ложным для того, чтобы итоговое утверждение было истинным. Например, если первое высказывание: \textit{Сегодня пятница}, а второе: \textit{Сегодня рабочий день}, то исключающее ИЛИ между ними будет истинным, если одно из высказываний истинно, а другое ложно.\\
Эту операцию также называют \textit{исключающей дизъюнкцией}. Она используется для проверки, что ровно одно из двух высказываний истинно. Исключающее ИЛИ записывается как \(A \oplus B\) или \(A \, XOR \, B\).\\
Логическая функция \((A \land \neg B) \lor (\neg A \land B)\) — запись исключающего ИЛИ через основные логические операции, т.е. исключающее ИЛИ можно заменить этим выражением, и функция не поменяет свои значения. Этим можно пользоваться при упрощении логического выражения.
Составим таблицу истинности для исключающего ИЛИ и его записи через стандартные операции:
\begin{figure}[h!]
    \begin{minipage}{0.45\linewidth}
    \centering
    \begin{tabular}{|c|c|c|}
    \hline
    $A$ & $B$ & $A \oplus B$ \\
    \hline
    0 & 0 & 0 \\
    0 & 1 & 1 \\
    1 & 0 & 1 \\
    1 & 1 & 0 \\
    \hline
    \end{tabular}
    \end{minipage}
    \hspace{0.5cm}
    \begin{minipage}{0.45\linewidth}
    \centering
    \begin{tabular}{|c|c|c|c|c|}
    \hline
    $A$ & $B$ & $\lnot A$ & $\lnot B$ & $(A \land \lnot B) \lor (\lnot A \land B)$ \\
    \hline
    0 & 0 & 1 & 1 & 0\\
    0 & 1 & 1 & 0 & 1\\
    1 & 0 & 0 & 1 & 1\\
    1 & 1 & 0 & 0 & 0\\
    \hline
    \end{tabular}
    \end{minipage}
\end{figure}
\subsubsection{Преобразования и правила}
TODO
\subsubsection{Булева логика в Python}
В Python существует тип переменной, предназначенный для работы с логическими значениями — \href{https://docs.python.org/3/library/stdtypes.html#boolean-type-bool}{булевый тип}. Переменные этого типа могут два значения: True (истина) и False (ложь). Поддерживается использование 1 вместо True и 0 вместо False.
\begin{minted}[linenos]{python}
a = True
b = 0
print(a, b)
\end{minted}
В таблице ниже приведены операторы для выполнения логических операций:
\begin{table}[h]
    \centering
    \begin{tabular}{c|c|c}
         Операция & Общая запись & Запись в Python \\ \hline
         Отрицание & \(\neg a\) & \texttt{not a} \\
         Конъюнкция & \(a \land b\) & \texttt{a and b} \\
         Дизъюнкция & \(a \lor b\) & \texttt{a or b} \\
         Импликация & \(a \Rightarrow b\) & \texttt{a <= b} \\
         Эквиваленция & \(a \Leftrightarrow b\) & \texttt{a == b} \\
         Исключающее ИЛИ & \(a \oplus b\) & \texttt{a != b} \\
         \hline
    \end{tabular}
    \caption{Логические операции в Python}
    \label{tab:lops_python}
\end{table}
Результат логических операций можно записывать в переменные, либо использовать в условии.
Пусть дана следующая функция \(F(a, b) = \neg a \land (a \lor \neg b)\), запишем ее в Python:
\begin{minted}[linenos]{python}
a = 0
b = 0
f = not a and (a or not b)
print(f)
\end{minted}
Значение функции будет выписано на экран. Приоритет операций сохраняется и в Python. \\
При более сложных выражений имеет смысл разделять все высказывание на части и записывать результаты в разные переменные. В конце достаточно их объединить самой последней операцией. Пример: \(F(x, y, z) = \neg x \Rightarrow ((\neg z \Leftrightarrow y) \lor (y \Rightarrow x))\). Разобьем на следующие части: \(A: \neg z \Leftrightarrow y\); \(B: y \Rightarrow x\); \(C: A \lor B\); \(F: \neg x \Rightarrow C\)
\begin{minted}[linenos]{python}
x = 1; y = 0; z = 1
a = not z == y
b = y <= x
c = a or b
f = not x <= c
print(f)
\end{minted}
\subsubsection{Перебор значений}
В данной задаче будет необходимо перебирать все возможные комбинации значений входных переменных.
При небольшом количестве переменных, это можно сделать с помощью вложенных циклов:
\begin{minted}[linenos]{python}
print(a, b, c)
for a in range(2):
    for b in range(2):
        for c in range(2):
            print(a, b, c)
\end{minted}
Работу данного кода можно представить в виде диаграммы:
\begin{figure}[h!]
    \centering
    \tikz {
        \node [circle, draw] (a) at (0,3) {a};
        \node [circle, draw] (b1) at (-4,2) {b};
        \node [circle, draw] (b2) at (4,2) {b};
        \node [circle, draw] (c1) at (-6,1) {c};
        \node [circle, draw] (c2) at (-2,1) {c};
        \node [circle, draw] (c3) at (2,1) {c};
        \node [circle, draw] (c4) at (6,1) {c};
        \node [circle, draw] (d1) at (-7,0) {000};
        \node [circle, draw] (d2) at (-5,0) {001};
        \node [circle, draw] (d3) at (-3,0) {010};
        \node [circle, draw] (d4) at (-1,0) {011};
        \node [circle, draw] (d5) at (1,0) {100};
        \node [circle, draw] (d6) at (3,0) {101};
        \node [circle, draw] (d7) at (5,0) {110};
        \node [circle, draw] (d8) at (7,0) {111};
        \graph [edges={thick}] {
            (a) -- (b1), (a) -- (b2),
            (b1) -- (c1), (b1) -- (c2), (b2) -- (c3), (b2) -- (c4),
            (c1) -- (d1), (c1) -- (d2), (c2) -- (d3), (c2) -- (d4),
            (c3) -- (d5), (c3) -- (d6), (c4) -- (d7), (c4) -- (d8),
        };
    }
\end{figure}
\subsection{Строки с пропущенными значениями}
В данных задачах, как правило, необходимо найти верное расположение переменных в таблице истинности, чтобы известные значения соответствовали им и давали правильный результат при подстановке в функцию.
\begin{problem}
Логическая функция F задаётся выражением \((x \lor y) \Rightarrow (z \equiv x)\).\\
Дан частично заполненный фрагмент, содержащий \textbf{неповторяющиеся} строки таблицы истинности функции F. Определите, какому столбцу таблицы истинности соответствует каждая из переменных x, y, z.
\begin{table}[ht]
\centering
\begin{tabular}{|c|c|c|c|}
    \hline
    \textbf{Переменная 1} & \textbf{Переменная 2} & \textbf{Переменная 3} & \textbf{Функция} \\ 
    \hline
    ??? & ??? & ??? & $F$ \\ 
    \hline
         & 0   & 0   & 0 \\ 
    \hline
         & 0   &     & 0 \\ 
    \hline
\end{tabular}
\end{table}
\end{problem}
\begin{solution}
Для решения задач этого типа будем использовать следующий шаблон:

\begin{minted}[linenos]{python}
def f(x, y, z): # как параметры передаем все переменные из выражения
    # Разделяем выражение на части и записываем в разные переменные
    t1 = x or y 
    t2 = z == x
    t3 = t1 <= t2
    return t3 # Конечное значение возвращаем

print("x", "y", "z")
for x in range(2): 
    for y in range(2):
        for z in range(2): # С помощью вложенных циклов перебираем значения
            if f(x, y, z) == 0: # Если нужный результат, выводим значения
                print(x, y, z)
\end{minted}
После запуска получаем:
\begin{minted}{python}
x y z
0 1 1
1 0 0
1 1 0
\end{minted}
Несложно заметить что только z дважды принимает значение 0, следовательно, 2 переменная - z. Когда z ноль, 3 переменная тоже обнуляется, следовательно, это y.\\
\textbf{Ответ:} \texttt{xzy}
\end{solution}
На что обратить внимание:
\begin{itemize}
    \item Чтобы не запутаться, выводим названия переменных (8-я строка) в таком же порядке, как и подходящую комбинацию (12-ая строка)
    \item Вложенных циклов должно быть столько, сколько и переменных в выражении. Для самопроверки можно вывести все значения без проверки, чтобы удостовериться в переборе всех возможных комбинаций
    \item Разделяем выражение на простейшие выражения без потери приоритетов (см. \autoref{tab:lops_priority_1})
\end{itemize}
\begin{problem}
Логическая функция F задаётся выражением \((x \land \neg y) \lor (y \equiv z) \lor \neg w\).\\ На рисунке приведён фрагмент таблицы истинности функции F, содержащий все наборы аргументов, при которых функция F ложна. Определите, какому столбцу таблицы истинности функции F соответствует каждая из переменных w, x, y, z. Все строки в представленном фрагменте разные.
\begin{table}[ht]
    \centering
    \begin{tabular}{|c|c|c|c|}
        \hline
        \textbf{Перем. 1} & \textbf{Перем. 2} & \textbf{Перем. 3} & \textbf{Перем. 4} \\
        \hline
        ??? & ??? & ??? & ??? \\
        \hline
         & 0 &  &  \\
        \hline
        1 & 0 &  & 0 \\
        \hline
        1 &  & 0 & 0 \\
        \hline
    \end{tabular}
\end{table}
\begin{solution}
Используем шаблон, сделав необходимые изменения:
\begin{minted}[linenos]{python}
def f(x, y, z, w): # как параметры передаем все переменные из выражения
    # Разделяем выражение на части и записываем в разные переменные
    t1 = x and not y
    t2 = y == z
    t3 = not w
    t4 = t1 or t2 or t3
    return t4 # Конечное значение возвращаем

print("x", "y", "z", "w")
for x in range(2): 
    for y in range(2):
        for z in range(2): 
            for w in range(2): # Добавляем еще один цикл, так как добавилась переменная
                if f(x, y, z, w) == 0: # Из условия — функция ложна
                    print(x, y, z, w)
\end{minted}
После запуска получаем:
\begin{minted}{python}
x y z w
0 0 1 1
0 1 0 1
1 1 0 1
\end{minted}
Так как в двух последних строках таблицы по два нуля, то первая строка в таблице может соответствовать только последнему выведенному ряду, следовательно, z - переменная 3. Переставим столбцы z и y и поменяем местами первую строку с последней, чтобы было проще сопоставлять:
\begin{minted}{python}
x z y w
0 0 1 1
1 0 1 1
0 1 0 1
\end{minted}
Тогда, очевидно, что w - переменная 1 (первый столбец содержит две единицы), поменяем его местами с x. Поменяем местами первые две строчки и получим соответствие:
\begin{minted}{python}
w z y x
1 0 1 1
1 0 1 0
1 1 0 0
\end{minted}
Обратите внимание, что выражение проверяем на ложность (14-ая строка) так как об этом сказано в условии (в прошлой задаче было указано в таблице): \textit{"приведён фрагмент таблицы истинности функции F, содержащий все наборы аргументов, при которых функция F \textbf{ложна}"}\\
\textbf{Ответ:} \texttt{wzyx}
\end{solution}
\end{problem}
\begin{problem}
Логическая функция F задаётся как $((x \land y) \lor (y \land z)) \equiv ((x \Rightarrow w) \land (w \Rightarrow z))$.\\
Дан частично заполненный фрагмент, содержащий неповторяющиеся строки таблицы истинности функции F. Определите, какому столбцу таблицы истинности соответствует каждая из переменных x, y, z, w.
\begin{table}[h]
    \centering
    \begin{tabular}{|c|c|c|c|c|}
        \hline
        \textbf{Переменная 1} & \textbf{Переменная 2} & \textbf{Переменная 3} & \textbf{Переменная 4} & \textbf{Функция} \\
        \hline
        ??? & ??? & ??? & ??? & \textit{F} \\
        \hline
        0 & 1 & 1 & 1 & 1 \\
        \hline
        0 & 1 & 0 &  & 1 \\
        \hline
        0 & 1 & 0 &  & 1 \\
        \hline
    \end{tabular}
\end{table}\\
В ответе напишите буквы x, y, z, w в том порядке, в котором идут соответствующие им столбцы (сначала — буква, соответствующая первому столбцу; затем — буква, соответствующая второму столбцу, и т.д.). Буквы в ответе пишите подряд, никаких разделителей между буквами ставить не нужно.
\begin{solution}
Используем шаблон, сделав необходимые изменения:
\begin{minted}[linenos]{python}
def f(x, y, z, w): # как параметры передаем все переменные из выражения
    # Разделяем выражение на части и записываем в разные переменные
    t1 = x and y
    t2 = y and z
    t3 = t1 or t2

    t4 = x <= w
    t5 = w <= z
    t6 = t4 and t5

    t7 = t3 == t6
    return t7 # Конечное значение возвращаем

print("x", "y", "z", "w")
for x in range(2): 
    for y in range(2):
        for z in range(2): 
            for w in range(2): # Добавляем еще один цикл, так как добавилась переменная
                if f(x, y, z, w) == 1: # Из условия — функция ложна
                    print(x, y, z, w)
\end{minted}
После запуска получаем:
\begin{minted}{python}
x y z w
0 0 0 1
0 1 0 1
0 1 1 0
0 1 1 1
1 0 0 0
1 0 0 1
1 0 1 0
1 1 1 1
\end{minted}
Так как строчка \texttt{0 1 1 1} - единственная, где три единицы, делаем вывод, что первая переменная правильно определена - x. Уберем все значения, где x = 1, так как таких нет в таблице:
\begin{minted}{python}
x y z w
0 1 1 1
0 0 0 1
0 1 0 1
0 1 1 0
\end{minted}
Переставим местами y и w:
\begin{minted}{python}
x w z y
0 1 1 1
0 1 0 0
0 1 0 1
0 0 1 1
\end{minted}
Таблица и вывод сопоставились.\\
\textbf{Ответ:} xwzy
\end{solution}
\end{problem}

\begin{problem}
Логическая функция F задаётся выражением:
\[(x \Rightarrow (y \equiv w)) \land (y \equiv (w \Rightarrow z)).\]
Дан частично заполненный фрагмент, содержащий неповторяющиеся строки таблицы истинности функции F.
Определите, какому столбцу таблицы истинности соответствует каждая из переменных x, y, z, w.
\begin{table}[H]
    \centering
    \begin{tabular}{|c|c|c|c|c|}
        \hline
        \textbf{Переменная 1} & \textbf{Переменная 2} & \textbf{Переменная 3} & \textbf{Переменная 4} & \textbf{Функция} \\
        \hline
         1 &  & 0 & 1 & 1 \\
        \hline
         0 & 0 &  & 0 & 1 \\
        \hline
         0 & 0 & 0 & 1 & 0 \\
        \hline
    \end{tabular}
\end{table}
В ответе напишите буквы x, y, z, w в том порядке, в котором идут соответствующие им столбцы (сначала — буква, соответствующая первому столбцу; затем  — буква, соответствующая второму столбцу, и т.д.). Буквы в ответе пишите подряд, никаких разделителей между буквами ставить не нужно.
\begin{solution}
В отличие от предыдущих задач данного типа, в таблице истинности значение функции меняется (последний столбец). В таких задачах, стоит разделить (условно) решение на две части: $F=1$ и $F=0$.\\
Выведем все комбинации, значение функции при которых 0, используя шаблон:
\begin{minted}[linenos, breaklines]{python}
def f(x, y, z, w): # как параметры передаем все переменные из выражения
    # Разделяем выражение на части и записываем в разные переменные
    t1 = y == w
    t2 = x <= t1
    
    t3 = w <= z
    t4 = y == t3
    
    t5 = t2 and t4
    return int(t5) # Конечное значение возвращаем как число
    # int(True) -> 1; int(False) -> 0; Возврат целого числа необходим для более понятной записи

print("x", "y", "z", "w", "f") # Добавим вывод значения функции
for x in range(2): 
    for y in range(2):
        for z in range(2): 
            for w in range(2): # Добавляем еще один цикл, так как добавилась переменная
                if f(x, y, z, w) == 0: # Из условия — функция ложна
                    print(x, y, z, w, f(x, y, z, w))
\end{minted}
Результат:
\begin{minted}[highlightlines={3, 6}]{python}
x y z w f
0 0 0 0 0
0 0 1 0 0
0 0 1 1 0
0 1 0 1 0
1 0 0 0 0
1 0 0 1 0
1 0 1 0 0
1 0 1 1 0
1 1 0 0 0
1 1 0 1 0
1 1 1 0 0
\end{minted}
Строки, которые теоретически могут подходить под третью строку таблицы выделим. Следовательно, на последней позиции может быть либо x, либо z. \textit{Предположим, что это z.}\\
Сделаем необходимые изменения в шаблоне и выведем комбинации при значении функции 1:
\begin{minted}[linenos, breaklines]{python}
def f(x, y, z, w): # как параметры передаем все переменные из выражения
    # Разделяем выражение на части и записываем в разные переменные
    t1 = y == w
    t2 = x <= t1
    
    t3 = w <= z
    t4 = y == t3
    
    t5 = t2 and t4
    return int(t5) # Конечное значение возвращаем как число
    # int(True) -> 1; int(False) -> 0; Возврат целого числа необходим для более понятной записи

print("x", "y", "w", "z", "f") # Добавим вывод значения функции
for x in range(2): 
    for y in range(2):
        for z in range(2): 
            for w in range(2): # Добавляем еще один цикл, так как добавилась переменная
                if f(x, y, z, w) == 1: # Из условия — функция ложна
                    print(x, y, w, z, f(x, y, z, w))
\end{minted}
Обратите внимание, что меняем местами переменные только в выводе на 13-ой строке и 19-ой (\textit{не меняем порядок параметров в функции ни при вызове, ни при декларации!}).
Результат:
\begin{minted}[highlightlines={2}]{python}
x y w z f
0 0 1 0 1
0 1 0 0 1
0 1 0 1 1
0 1 1 1 1
1 1 1 1 1
\end{minted}
Заметим, что выделенная строка, подходит под 2 строку из таблицы, однако отсутствует первая строка из таблицы. Так как, z (согласно предположению) стоит на правильном месте, а w может стоять только на 3 позиции (иначе не подходит), то единственным вариантом перестановки является замена x и y (строка останется неизменной, но возможно будет найдена первая строка из таблицы):
\begin{minted}[highlightlines={4}]{python}
y x w z f
0 0 1 0 1
1 0 0 0 1
1 0 0 1 1
1 0 1 1 1
1 1 1 1 1
\end{minted}
Нашлась первая строка из таблицы, а значит такая последовательность нам подходит (изначальное предположение — верно).\\
\textbf{Ответ:} yxwz
\end{solution}
\end{problem}
В этом подтипе часто встречается необходимость в предположениях. В самом начале решения, мы увидели, что последнюю позицию должна занимать одна из двух возможных переменных. В таком случае, стоит сделать предположение и продолжить решение. Если таблица будет соответствовать данной, значит предположение — верно, в противном случае, предположение было неверно и другой вариант должен быть рассмотрен.\\
Еще одной особенностью данного решения является поэтапный поиск строк. Сделав предположение, мы "нашли" одну из трех строк. Посмотрев на другое значение функции, мы "нашли" следующую строку. Изменили порядок и "нашли" последнюю строку. Главное, в таком процессе не "находить" новые строки, "теряя" ранее полученные.

\begin{problem}
Две логические функции заданы выражениями:
\begin{gather*}
F1 = (x \Rightarrow y) \equiv (w \lor \neg z)\\
F2 = (x \Rightarrow y) \land (\neg w \equiv z)
\end{gather*}
Дан частично заполненный фрагмент, содержащий неповторяющиеся строки таблицы истинности обеих функций. Определите, какому столбцу таблицы истинности соответствует каждая из переменных w, x, y, z:
\begin{table}[h]
    \centering
    \begin{tabular}{|c|c|c|c||c|c|}
        \hline
        ??? & ??? & ??? & ??? & $F_1$ & $F_2$ \\
        \hline
         & 1 & 0 & 1 &  & 0 \\
        \hline
         & 0 & 0 & 0 & 0 &  \\
        \hline
        0 &  & 0 & 0 & 0 & 1 \\
        \hline
    \end{tabular}
\end{table}
\begin{solution}
Используем шаблон, сделав необходимые изменения:
\begin{minted}[linenos]{python}
def f1(x, y, z, w): # как параметры передаем все переменные из выражения
    # Разделяем выражение на части и записываем в разные переменные
    t1 = x <= y
    t2 = w or not z
    t3 = t1 == t2
    return int(t3) # Конечное значение возвращаем

def f2(x, y, z, w): # как параметры передаем все переменные из выражения
    # Разделяем выражение на части и записываем в разные переменные
    t1 = x <= y
    t2 = not w == z
    t3 = t1 and t2
    return int(t3) # Конечное значение возвращаем

print("x", "y", "z", "w", "#", "1", "2")
for x in range(2): 
    for y in range(2):
        for z in range(2): 
            for w in range(2): # Добавляем еще один цикл, так как добавилась переменная
                print(x, y, z, w, '|', f1(x, y, z, w), f2(x, y, z, w))
\end{minted}
После запуска получаем:
\begin{minted}{python}
x y z w # 1 2
0 0 0 0 | 1 0
0 0 0 1 | 1 1
0 0 1 0 | 0 1
0 0 1 1 | 1 0
0 1 0 0 | 1 0
0 1 0 1 | 1 1
0 1 1 0 | 0 1
0 1 1 1 | 1 0
1 0 0 0 | 0 0
1 0 0 1 | 0 0
1 0 1 0 | 1 0
1 0 1 1 | 0 0
1 1 0 0 | 1 0
1 1 0 1 | 1 1
1 1 1 0 | 0 1
1 1 1 1 | 1 0
\end{minted}
Заметим, что среди тех строчек, где F1 и F2 принимают значения \texttt{0 1}, только одна подходит:
\texttt{0 0 1 0 | 0 1}. Следовательно, переменная 2 - z. Переставим местами два столбца в середине и уберем строки, где F1 и F2 не образуют нужных комбинаций или исходные переменные не могут быть теоретически использованы:
\begin{minted}[highlightlines={2, 3, 6}]{python}
x z y w # 1 2
0 1 0 0 | 0 1
0 1 0 1 | 1 0
0 0 1 0 | 1 0
0 1 1 1 | 1 0
1 0 0 0 | 0 0
1 1 0 0 | 1 0
1 0 1 0 | 1 0
1 1 1 1 | 1 0
\end{minted}
Выделенные строки соответствуют данной таблице.\\
\textbf{Ответ:} zxyw
\end{solution}
При решении данного типа, часто не имеет смысл накладывать условия на F1 и F2. Вместо этого стоит самостоятельно вычеркивать линии, которые не подходят никаким образом (например, в данной таблице все строки содержат хотя бы 2 единицы, значит из вывода можно убрать все строки, где единиц нет совсем или только одна).
\end{problem}
\newpage

\section{Задание 3. Поиск информации в реляционных базах данных}
\subsection{Спецификация}
\textbf{Задание проверяет:}
Умение поиска информации в реляционных базах данных.\\
\textbf{Уровень сложности:}
Базовый\\
\textbf{Максимальный балл:}
1\\
\textbf{Примерное время:}
3 минуты (2,5 минуты)

\subsection{Теория}
\begin{definition}
База данных (БД) — это имеющая название совокупность данных, которая отражает состояние объектов и их отношений в рассматриваемой предметной области. 
\end{definition}
Как правило, БД представлена в виде таблицы, где одна строка соответствует одной записи. Каждый столбец отвечает за какое-то конкретное значение, определенное для записей. Пример базы данных, представленной в виде таблицы:
\begin{table}[!h]
    \centering
    \begin{tabular}{|c|c|c|c|c|c|} \hline
        ID & Имя & Фамилия & Рост & Школа & Кол-во одноклассников \\ \hline
        29 & Степан & Сидоров & 167 & ГБОУ №1010 & 23 \\ \hline
        30 & Анна & Петрова & 160 & ГБОУ №2345 & 25 \\ \hline
        31 & Иван & Иванов & 175 & ГБОУ №678 & 22 \\ \hline
        32 & Мария & Смирнова & 162 & ГБОУ №345 & 27 \\ \hline
        33 & Алексей & Кузнецов & 170 & ГБОУ №9876 & 20 \\ \hline
    \end{tabular}
    \caption{Пример БД (таблица)}
    \label{tab:db_examp_as_table}
\end{table}

Мы будем работать с базами данных, которые представлены в виде таблицы в формате .xlsx (Excel таблица). Данный файл уже будет загружен на компьютер, для работы с ним понадобиться редактор электронных таблиц. На экзамене хотя бы один будет уже предустановлен (скорее всего \href{https://myoffice.ru/apps/table/}{«МойОфис Таблица»} или \href{https://www.microsoft.com/ru-ru/microsoft-365/excel?market=ru}{Microsoft Excel}). Функционал большинства редакторов электронных таблиц похож, однако могут встречаться свои особенности. В данном блоке теории будет использоваться редактор \href{https://www.microsoft.com/ru-ru/microsoft-365/excel?market=ru}{Microsoft Excel}.\\
\subsubsection{Знакомство с редактором таблиц}
Открыть файл можно либо двойным кликом по файлу с БД или открытием непосредственно в приложении редактора (в Excel: Открыть $\to$ Обзор $\to$ "находим файл в файловой системе"). 
\begin{figure}[H]
    \centering
    \includegraphics[width=0.75\linewidth]{3_source/3_mainview.png}
    \caption{Главный вид}
    \label{fig:excel_mainview}
\end{figure}
\textcolor{red}{Красным} прямоугольником обозначено "имя" выделенной ячейки, изменять значение этой клетки можно в поле, выделенном \textcolor{yellow}{желтым} цветом. Имя ячейки используется как уникальный идентификатор и складывается из имени столбца (выделено \textcolor{violet}{фиолетовым}) и номера строчки (выделено \textcolor{orange}{оранжевым}). Снизу, в белой рамке выделена таблица, на который мы находимся и ее имя - в данном примере это \textit{Ученики}.\\
Несколько ячеек можно выделять для копирования/вставки. Все стандартные комбинации работают и в этих редакторах: \texttt{ctrl + C}; \texttt{ctrl + V}; \texttt{ctrl + Z}; \texttt{ctrl + A}; Для выделения столбца или строчки, кликаем на соответствующее место в нумерации/наименовании строк/столбцов.\\ 
Редактирование стилизации ячеек можно осуществлять на панели инструментов (набор функциональных кнопок и меню в верхней части экрана), на вкладке \textit{Главная}. В данном блоке теории, касательно стилизации, стоит лишь упомянуть, что размеры столбов и строк можно изменять, двигая ползунок, появляющийся при наведении на границу двух строк/столбцов. Если какая-то информация, записанная в ячейке не видна из-за размеров ячейки, то ее стоит увеличить.
\subsubsection{Формулы}
Представим, что необходимо в какую-то ячейку записать средний рост всех учеников из таблицы. Как мы знаем, среднее значение вычисляется по формуле $(\sum_{i=0}^{n} x_i) / n$, то есть, необходимо посчитать сумму всех величин из столбца роста. Для небольшой таблицы, это сделать несложно, однако что если количество записей исчисляется тысячами? А если мы уберем пару записей, сумму нужно будет пересчитывать? Чтобы автоматизировать этот процесс, большинство редакторов  электронных таблиц предлагают использовать формулы (функции).
\begin{definition}
Формула (функция) в редакторе электронных таблиц — это выражение, которое выполняет вычисления на основе данных, введённых в ячейки.
\end{definition}
Формулы могут включать числа, ссылки на другие ячейки, арифметические операторы и встроенные функции. Результат вычисления формулы автоматически отображается в ячейке, где формула была введена, и обновляется при изменении данных, на которые она ссылается. В большинстве редакторов, запись функции начинается со знака равно.\\
Попробуем записать формулу, для подсчета суммы значений в некоторых ячейках:
\begin{figure}[H]
    \centering
    \includegraphics[width=1\linewidth]{3_source/3_function.png}
    \caption{Запись функции суммы}
    \label{fig:excel_sum}
\end{figure}
Обратите внимание, что каждая функция имеет свое собственное название (всегда заглавными буквами) и синтаксис. В \autoref{fig:excel_sum}, функция имеет название \texttt{СУММ}, ее параметры  — имена ячеек или константы (перечисляются через ";"), итоговое значение — сумма параметров.\\
В данном блоке теории не будет описана каждая используемая формула, но только те, чей результат и механизм работы не является очевидным. В \autoref{tab:excel_gen_functions} приведены основные формулы, которые мы будем использовать.
\begin{table}[H]
    \centering
    \begin{tabular}{|c|c|p{5cm}|} \hline
         Название & Пример & Описание \\ \hline
         \texttt{СУММ} & \texttt{=СУММ(A2:A10;C2:C10)} & Складывает значения ячеек A2:10, а также ячеек C2:C10 (\href{https://support.microsoft.com/ru-ru/office/%D1%81%D1%83%D0%BC%D0%BC-%D1%84%D1%83%D0%BD%D0%BA%D1%86%D0%B8%D1%8F-%D1%81%D1%83%D0%BC%D0%BC-043e1c7d-7726-4e80-8f32-07b23e057f89}{Документация})\\ \hline
         \texttt{ПРОИЗВЕД} & \texttt{=ПРОИЗВЕД(A2:A4; 2)} & Перемножает числа из ячеек A2–A4, а затем умножает полученный результат на 2 (\href{https://support.microsoft.com/ru-ru/office/%D1%84%D1%83%D0%BD%D0%BA%D1%86%D0%B8%D1%8F-%D0%BF%D1%80%D0%BE%D0%B8%D0%B7%D0%B2%D0%B5%D0%B4-8e6b5b24-90ee-4650-aeec-80982a0512ce}{Документация})\\ \hline
         \texttt{ЧАСТНОЕ} & \texttt{=ЧАСТНОЕ(-10;A3)} & Целая часть результата деления -10 на значение ячейки A3 (\href{https://support.microsoft.com/ru-ru/office/%D1%87%D0%B0%D1%81%D1%82%D0%BD%D0%BE%D0%B5-%D1%84%D1%83%D0%BD%D0%BA%D1%86%D0%B8%D1%8F-%D1%87%D0%B0%D1%81%D1%82%D0%BD%D0%BE%D0%B5-9f7bf099-2a18-4282-8fa4-65290cc99dee}{Документация})\\ \hline
         \texttt{КОРЕНЬ} & \texttt{=КОРЕНЬ(A2)} & Квадратный корень значения ячейки А2 (\href{https://support.microsoft.com/ru-ru/office/%D1%87%D0%B0%D1%81%D1%82%D0%BD%D0%BE%D0%B5-%D1%84%D1%83%D0%BD%D0%BA%D1%86%D0%B8%D1%8F-%D1%87%D0%B0%D1%81%D1%82%D0%BD%D0%BE%D0%B5-9f7bf099-2a18-4282-8fa4-65290cc99dee}{Документация})\\ \hline
         \texttt{ЕСЛИ} & \texttt{=ЕСЛИ(A1>10;"Больше";"Не превосходит")} & Если значение A1 больше 10, записать в данную ячейку \textit{Больше}, иначе записать \textit{Не превосходит} (\href{https://support.microsoft.com/ru-ru/office/%D0%B5%D1%81%D0%BB%D0%B8-%D1%84%D1%83%D0%BD%D0%BA%D1%86%D0%B8%D1%8F-%D0%B5%D1%81%D0%BB%D0%B8-69aed7c9-4e8a-4755-a9bc-aa8bbff73be2}{Документация})\\ \hline
         \texttt{СЧЁТ} & \texttt{=СЧЁТ(A5:A7)} & Подсчитывает количество ячеек, содержащих числа, в диапазоне A5:A7 (\href{https://support.microsoft.com/ru-ru/office/%D1%84%D1%83%D0%BD%D0%BA%D1%86%D0%B8%D1%8F-%D1%81%D1%87%D1%91%D1%82-a59cd7fc-b623-4d93-87a4-d23bf411294c}{Документация})\\ \hline
         \texttt{СЧЁТЕСЛИ} & \texttt{=СЧЁТЕСЛИ(B2:B5;'>55')} & Количество ячеек со значением больше 55 в ячейках В2–В5 (\href{https://support.microsoft.com/ru-ru/office/%D1%81%D1%87%D1%91%D1%82%D0%B5%D1%81%D0%BB%D0%B8-%D1%84%D1%83%D0%BD%D0%BA%D1%86%D0%B8%D1%8F-%D1%81%D1%87%D1%91%D1%82%D0%B5%D1%81%D0%BB%D0%B8-e0de10c6-f885-4e71-abb4-1f464816df34}{Документация})\\ \hline
         \texttt{СУММЕСЛИ} & \texttt{=СУММЕСЛИ(A2:A5;'>100';B2:B5)} & Ячейки из B2:B5 является слагаемым, если соответсвующая ячейка из A2:A5 имеет значение больше 100 (\href{https://support.microsoft.com/ru-ru/office/%D1%84%D1%83%D0%BD%D0%BA%D1%86%D0%B8%D1%8F-%D1%81%D1%83%D0%BC%D0%BC%D0%B5%D1%81%D0%BB%D0%B8-169b8c99-c05c-4483-a712-1697a653039b}{Документация})\\ \hline
         \texttt{И} & \texttt{=И(A2>1;A2<100)} & ИСТИНА, если число в ячейке A2 больше 1 \textbf{И} меньше 100, иначе ЛОЖЬ (\href{https://support.microsoft.com/ru-ru/office/%D1%84%D1%83%D0%BD%D0%BA%D1%86%D0%B8%D1%8F-%D0%B8-5f19b2e8-e1df-4408-897a-ce285a19e9d9}{Документация})\\ \hline
         \texttt{ИЛИ} & \texttt{=И(A2>1;A2<100)} & ИСТИНА, если число в ячейке A2 больше 1 \textbf{ИЛИ} меньше 100, иначе ЛОЖЬ (\href{https://support.microsoft.com/ru-ru/office/%D0%B8%D0%BB%D0%B8-%D1%84%D1%83%D0%BD%D0%BA%D1%86%D0%B8%D1%8F-%D0%B8%D0%BB%D0%B8-7d17ad14-8700-4281-b308-00b131e22af0}{Документация})\\ \hline
         \texttt{НЕ} & \texttt{=НЕ(A2>100)} & A2 НЕ больше 100 (\href{https://support.microsoft.com/ru-ru/office/%D1%84%D1%83%D0%BD%D0%BA%D1%86%D0%B8%D1%8F-%D0%BD%D0%B5-9cfc6011-a054-40c7-a140-cd4ba2d87d77}{Документация})\\ \hline
    \end{tabular}
    \caption{Основные функции в Excel}
    \label{tab:excel_gen_functions}
\end{table}
Это лишь основные операции, которые будут встречаться в решениях чаще всего. Однако существует много и других функций, которые будут использоваться, с ними необходимо познакомиться во время решения. На \autoref{fig:excel_mainview}, слева от области, выделенной желтым цветом располагается кнопка \textit{Fx} - вставка функции. При нажатии открывается меню, в которой можно найти необходимую функцию и небольшую справку к ней.
\subsubsection{Реляционные базы данных}
При выполнении работ в школе, мы часто подписываемся с помощью имени и фамилии. Не сложно догадаться, что использование лишь имени усложнило бы процесс идентификации ученика при проверке работы, так как нередко имена в одном классе повторяются. Поэтому, комбинация \textit{имя + фамилия} в контексте школьных работ — хороший идентификатор, так как почти всегда является \textit{уникальным}.
\begin{definition}
Идентификатор (ключ) – это значение, которое однозначно определяет запись. Так же обозначается как \textit{ID} (от англ. identification)
\end{definition}
Однако при заполнении документов в МФЦ, указывается серия и номер паспорта, потому что комбинация \textit{имя + фамилия} может легко совпасть у двух разных людей в контексте района. В данном случае серия и номер паспорта — хороший идентификатор, так как является уникальным для каждого гражданина (уникальность обеспечивает контролируемая выдача идентификаторов, которая исключает вероятность коллизии).
\begin{definition}
Коллизия — ситуация, при которой идентификаторы у двух записей совпадают.
\end{definition}
Теперь представим, что у нас есть БД (таблица), с записями о школьниках и их средних баллах за три предмета:
\begin{table}[H]
    \centering
    \begin{tabular}{|c|c|c|c|c|} \hline
        ID & Имя & Фамилия & Математика & Информатика \\ \hline
        29 & Степан & Сидоров & 3.78 & 3.75 \\ \hline
        30 & Анна & Петрова & 4.00 & 3.50 \\ \hline
        31 & Иван & Иванов & 4.70 & 4.80 \\ \hline
        32 & Мария & Смирнова & 5.00 & 4.90 \\ \hline
        33 & Алексей & Кузнецов & 3.66 & 3.80 \\ \hline
    \end{tabular}
    \caption{Записи об учениках и их оценках}
    \label{tab:relational_db_1}
\end{table}
Наверняка, школьник захочет проверить правильность внесения данных в данную таблицу. Значит, должна быть другая БД (таблица), с оценками за каждую работу, для каждого ученика. Однако \autoref{tab:relational_db_1} уже содержит информацию о том, какой ID присвоен какому ученику. Соответственно, в таблице по отдельному предмету нет необходимости в такой информации, ученика будем идентифицировать только по ID. Посмотрим на эту таблицу:
\begin{table}[H]
    \centering
    \begin{tabular}{|c|c|c|c|c|} \hline
        ID Ученика & СР №1 & ДЗ №1 & СР №2 & КР №1 \\ \hline
        29 & 4 & 5 & 3 & 3 \\ \hline
        30 & 4 & 4 & 4 & 4 \\ \hline
        31 & 5 & 4 & 3 & 4 \\ \hline
        32 & 3 & 4 & 3 & 5 \\ \hline
        33 & 5 & 5 & 4 & 5 \\ \hline
    \end{tabular}
    \caption{Записи о работах по Информатике}
    \label{tab:relational_db_2}
\end{table}
Таким образом, можно сказать, что \autoref{tab:relational_db_1} и \autoref{tab:relational_db_2} связаны между собой (имея ID школьника из \autoref{tab:relational_db_1} можем посмотреть его оценки из \autoref{tab:relational_db_2}, и наоборот). Тогда, можно объединить эти две БД (таблицы) в одну БД — такая база данных называется реляционной.
\begin{definition}
Реляционная база данных – это совокупность таблиц, которые связаны между собой. Связь создается с помощью идентификаторов (ключевых полей).
\end{definition}
Слово реляция основано от англ. relation (отношение, зависимость, связь). Часто, для репрезентации реляционных БД используются диаграммы, где поля, отвечающие за одно и то же значение, в разных таблицах соединенны линиями - \autoref{fig:relational_db_uml}.
\begin{figure}[!ht]
\centering
\begin{circuitikz}[scale=0.8, transform shape]
\tikzstyle{every node}=[font=\large]
\draw  (6,10.5) rectangle  node {\large ID} (10,9.5);
\draw  (6,9.5) rectangle  node {\large Имя} (10,8.5);
\draw  (6,8.5) rectangle  node {\large Фамилия} (10,7.5);
\draw  (6,7.5) rectangle  node {\large Математика} (10,6.5);
\draw  (6,6.5) rectangle  node {\large Информатика} (10,5.5);

\draw  (13,10.5) rectangle  node {\large ID Ученика} (16,9.5);
\draw  (13,9.5) rectangle  node {\large СР №1} (16,8.5);
\draw  (13,8.5) rectangle  node {\large ДЗ №1} (16,7.5);
\draw  (13,7.5) rectangle  node {\large СР №2} (16,6.5);
\draw  (13,6.5) rectangle  node {\large КР №1} (16,5.5);
\draw [short] (10,10) -- (13,10);
\end{circuitikz}
\caption{Реляция \autoref{tab:relational_db_1} и \autoref{tab:relational_db_2}}
\label{fig:relational_db_uml}
\end{figure}\\
Каждая задача данного типа подразумевает умение работать с такими таблицами. При этом количество реляций в одной БД может быть больше 1 (зачастую 2).
\subsubsection{Фильтры}
Представим, что в БД находится 30 записей об учениках и их итоговых оценках за три предмета. Наша задача: выбрать в олимпиадный класс по информатике 10 учеников, чьи итоговые оценки за этот предмет - 4 или 5. Мы бы могли проходиться по каждой записи и смотреть, отвечает ли она нашим критериям. Если ученик подходит, запомним его ID.\\
Представим, что мы решили отбирать учеников еще и по итоговой оценке с математики. Тогда, нам придется пройтись еще раз по записям и уже отобрать по двум критериям. Такой подход займет сколько-то времени, но по итогу у нас будут необходимые нам записи.\\
Если количество записей будет исчисляться сотнями, то описанный ранее способ уже не подойдет: будет потрачено слишком много времени. Чтобы автоматизировать данную задачу, большинство редакторов электронных таблиц имеют функцию фильтрации.
\begin{definition}
Фильтр (в эл. таблицах) - инструмент, позволяющий отображать только те записи, которые отвечают заданным критериям.
\end{definition}
В данном тексте будет приведен небольшой пример использования фильтров (рекомендуется ознакомиться и с \href{https://support.microsoft.com/ru-ru/office/%D1%84%D0%B8%D0%BB%D1%8C%D1%82%D1%80%D0%B0%D1%86%D0%B8%D1%8F-%D0%B4%D0%B0%D0%BD%D0%BD%D1%8B%D1%85-%D0%B2-%D0%B4%D0%B8%D0%B0%D0%BF%D0%B0%D0%B7%D0%BE%D0%BD%D0%B5-%D0%B8%D0%BB%D0%B8-%D1%82%D0%B0%D0%B1%D0%BB%D0%B8%D1%86%D0%B5-7fbe34f4-8382-431d-942e-41e9a88f6a96}{документацией})
Чтобы воспользоваться фильтром в таблице, его необходимо сначала активировать, это можно сделать либо комбинацией клавиш \texttt{CTRL + SHIFT + L}, либо нажатием на кнопку Фильтр на панели инструментов Данные.
\subsection{Задания для подготовки}
\newcommand\PThreeSource{./3\_problems}
\newcommand{\fileref}[1]{\href{run:\PThreeSource/#1}{\texttt{#1}}}

\textit{Все файлы с которыми мы будем работать в данном разделе находятся в \href{run:\PThreeSource}{\texttt{\PThreeSource}}}.
\begin{problem}
В \fileref{3\_1.xlsx} приведён фрагмент базы данных «Продукты» о поставках товаров в магазины районов города. База данных состоит из трёх таблиц.\\
Таблица «Движение товаров» содержит записи о поставках товаров в магазины в течение первой декады июня 2021 г., а также информацию о проданных товарах. Поле Тип операции содержит значение Поступление или Продажа, а в соответствующее поле Количество упаковок, шт. занесена информация о том, сколько упаковок товара поступило в магазин или было продано в течение дня. На рисунке приведена схема указанной базы данных:
\begin{figure}[H]
    \centering
    \includegraphics[scale=0.5]{3_source/3_1.png}
\end{figure}
Используя информацию из приведённой базы данных, определите, на сколько увеличилось количество упаковок яиц диетических, имеющихся в наличии в магазинах Заречного района за период с 1 по 10 июня.\\
В ответе запишите только число.
\begin{solution}

\end{solution}
\end{problem}

\newpage
\section{Задание 14. Кодирование чисел. Системы счисления}
\textbf{Задание проверяет:}
Знания касающееся позиционных систем счисления.\\
\textbf{Уровень сложности:}
Повышенный\\
\textbf{Максимальный балл:}
1\\
\textbf{Примерное время:}
3 минуты (3 минуты)

\subsection{Теория}
\begin{definition}
Система счисления — это знаковая система, в которой приняты определённые правила записи чисел.    
\end{definition}
Знаки, с помощью которых записываются числа, называются цифрами, а их совокупность — алфавитом системы счисления. Для удобства, будем использовать сокращение СС для системы счисления.\\
Важно понимать, что СС определяет лишь запись числа, а не его значение (количественный эквивалент), так Римляне, увидев четырнадцать барашков запишут: \MakeUppercase{\romannumeral 14}, а мы запишем $14$. Несмотря на то, что записи отличаются, мы говорим об одном и том же количественном эквиваленте.\\
Можно выделить три типа систем счисления:
\begin{itemize}
    \item Унарная
    \item Непозиционная
    \item Позиционная   
\end{itemize}
\subsubsection{Унарная СС}
\begin{definition}
Унарная СС — система счисления, в которой используется лишь одна цифра.
\end{definition}
Количество записанных цифр и есть само значение числа. Например, если $1$ — цифра в унарной СС, то число $1111 = 4$. Такая СС примитивно простая, однако неудобна при выполнении арифметических операций, записи дробей и тд. С данной системой счисления мы работать не будем.
\subsubsection{Непозиционная CC}
\begin{definition}
Непозиционная CC — система счисления, в которой количественный эквивалент (количественное значение) цифры в числе \textbf{не} зависит от её положения в записи числа.
\end{definition}
В большинстве непозиционных систем счисления числа образуются путём сложения "узловых" чисел (числа запись которых определена заранее, т.е. получена неалгоритмически).\\
Примером непозиционной СС служит римская система счисления:
\clearpage
\begin{table}[!htb]
    \centering
    \begin{tabular}{c|c} 
        Римская запись & Значение \\ \hline
        \MakeUppercase{\romannumeral 1} & 1 \\
        \MakeUppercase{\romannumeral 5} & 5 \\
        \MakeUppercase{\romannumeral 10} & 10 \\
        \MakeUppercase{\romannumeral 50} & 50 \\
        \MakeUppercase{\romannumeral 100} & 100 \\
        \MakeUppercase{\romannumeral 500} & 500 \\
        \MakeUppercase{\romannumeral 1000} & 1000 \\
    \end{tabular}
    \caption{Римские цифры}
    \label{tab:roman_nums}
\end{table}
Числа получаются путём сложения и вычитания "узловых" чисел с учётом следующего правила: каждый меньший знак, поставленный справа от большего, прибавляется к его значению, а каждый меньший знак, поставленный слева от большего, вычитается из него. Например, $17 = $ \MakeUppercase{\romannumeral 17}, а $64 = $ \MakeUppercase{\romannumeral 64}\\
Задание данного типа не подразумевает работу с такими СС.
\subsubsection{Позиционная CC}
\begin{definition}
Позиционная CC — система счисления, в которой количественный эквивалент (количественное значение) цифры в числе зависит от её положения в записи числа.
\end{definition}
Для позиционных СС определено понятие основания СС.
\begin{definition}
Основание позиционной системы счисления — количество уникальных цифр, составляющих её алфавит.
\end{definition}
Основанием СС может служить любое натуральное число $q > 1$. Алфавитом произвольной позиционной системы счисления с основанием q служат числа $О$, $1$, ..., $q-1$, каждое из которых может быть записано с помощью одного уникального символа; младшей цифрой всегда является О. Обратите внимание, что максимальная цифра в записи числа всегда меньше основания на 1.\\
В позиционной СС с основанием q любое число может быть представлено в виде:
\begin{equation}
A_q = \pm (a_{n-1}*q^{n-1} + a_{n-2}*q^{n-2} + ... + a_{0}*q^{0}) \label{eq:pos_nums_1}
\end{equation}
В данной записи:\\
$A_q$ - число (q говорит о записи в СС с основанием q);\\
$q$ - основание СС;\\
$n$ — количество целых разрядов числа;\\
$a_i$ - цифры числа, стоящая на i-ом разряде;\\
$q^i$ - "вес" i-го разряда;\\
Обратите внимание, что нумерация разряда идет справа налево, начиная с 0.\\
Запись можно расширить для определения и дробных чисел:
\begin{equation}
A_q = \pm (a_{n-1}*q^{n-1} + a_{n-2}*q^{n-2} + ... + a_{0}*q^{0} + a_{-1}*q^{-1} + a_{-2}*q^{-2} + ... + a_{-m}*q^{-m}) \label{eq:pos_nums_2}
\end{equation}
Здесь, $m$ - количество дробных разрядов числа.\\
Большинство чисел, которые нас окружают — числа записанные в позиционной СС с основанием 10.\\ Как пример, возьмем число 3425 и запишем его c помощью \autoref{eq:pos_nums_1}:
\[
\textcolor{red}{3}\textcolor{orange}{4}\textcolor{green}{2}\textcolor{blue}{5}_{10} = \textcolor{red}{3} * 10^3 + \textcolor{orange}{4} * 10^2 + \textcolor{green}{2} * 10^1 + \textcolor{blue}{5} * 10^0 = 3000 + 400 + 20 + 5
\]
Можно проделать то же самое и для дробного числа 3425,97 с помощью \autoref{eq:pos_nums_2}:
\[
3425,97_{10} = 3 * 10^3 + 4 * 10^2 + 2 * 10^1 + 5 * 10^0 + 9 * 10^{-1} + 7 * 10^{-2} = 300 + 400 + 20 + 5 + 0,9 + 0,07
\]
Обратите внимание, что запись числа в виде $A_i$ в данном контексте, говорит, что оно записано в системе счисления с основанием $i$. Так как в данных задачах используются только позиционные СС, то только о них мы и будем говорить, при этом слово позиционная будет опускаться.\\
Посмотрим на число не из десятичной СС, например: $11011_2$. Систему счисления с основанием 2 обычно называют \textit{бинарной} или \textit{двоичной} СС.
\subsubsection{Системы счисления с основанием больше 10}
Заметим, что максимальная цифра в СС с основанием десять - 9. Возникает логичный вопрос, какая максимальная цифра будет в СС с основанием 11? В количественном эквиваленте (т.е. по значению) это действительно 10, однако запись $10_{11}$ можно тогда расценить и как запись числа с двумя цифрами (1 и 0), так и как запись числа с одной цифрой (10). Чтобы решить такую ситуацию стали применять заглавные латинские буквы в записи числа. Соответсвующее каждой букве значения приведены в \autoref{tab:latin_digits}:
\begin{table}[!h]
    \centering
    \begin{tabular}{c|c}
        Запись & Эквивалент \\ \hline
        A & 10 \\
        B & 11 \\
        C & 12 \\
        D & 13 \\
        E & 14 \\
        F & 15 \\
    \end{tabular}
    \caption{Запись цифр через латинские буквы}
    \label{tab:latin_digits}
\end{table}\\
Теперь мы умеем различать оба варианта: $A_{11}$ и $10_{11}$. \\
Как выглядит запись максимального трехзначного числа в шестнадцатеричной СС? ($FFF_{16}$).\\
Обязательным навыком для решения данных задач является перевод из одной СС в другую СС. Разобьем 
эту операцию на две: $A_i \Rightarrow B_{10}$ и $B_{10} \Rightarrow C_j$. Будем проделывать эти действия используя строгие алгоритмы.
\subsubsection{Перевод из n-ой системы счисления в 10-ую}\label{sec:base_n_to_base_10}
По сути, этот перевод мы уже пару раз сделали используя \autoref{eq:pos_nums_1}. Выражая какое-то число по этой формуле, мы в процессе вычислений уже используем 10-чную СС, соответственно и результат получим в 10-ной СС, для примера, переведем число $1342_5$:
\begin{enumerate}
    \item Пронумеруем разряды числа справа налево, начиная с 0: $\underset{3}{1} \underset{2}{3} \underset{1}{4} \underset{0}{2}$;
    \item Согласно нумерации выразим число через \autoref{eq:pos_nums_1};
\end{enumerate}
\[
\underset{\textcolor{red}{3}}{1} \underset{\textcolor{blue}{2}}{3} \underset{\textcolor{orange}{1}}{4} \underset{\textcolor{green}{0}}{2} = 1 * 5^{\textcolor{red}{3}} + 3 * 5^{\textcolor{blue}{2}} + 4 * 5^{\textcolor{orange}{1}} + 2 * 5^{\textcolor{green}{0}} = 1 * 125 + 3 * 25 + 4 * 5 + 2 * 1 = 125 + 75 + 20 + 2 = 222_{10} 
\]
Таким образом, $1342_5 = 222_{10}$. Как правило, визуально число должно уменьшится, если перевод выполнялся из СС с основанием меньшим 10, и наоборот (так как вес каждого разряда увеличивается/уменьшается).\\
Перевод вещественного числа осуществляется с помощью \autoref{eq:pos_nums_2}:
\[
123,22_{4} = 1 * 4^2 + 2 * 4^1 + 3 * 4^0 + 2 * 4^{-1} + 2 * 4^{-2} = 1 * 16 + 2 * 4 + 3 * 1 + 2 * 0,25 + 2 * 0,0625 
\]
Часто будем пользоваться переводом из бинарной СС в 10-чную, при таком переводе прибегать к полной записи нет необходимости, ведь если на соответствующем месте стоит 0, значит разряд просто не учитывается, если стоит 1, то значение разряда - $2^i$:
\[
\underset{5}{1} \underset{4}{0} \underset{3}{1} \underset{2}{0} \underset{1}{1} \underset{0}{1} = 2^5 + 2^3 + 2^1 + 2^0 = 32 + 8 + 2 + 1 = 43_{10}
\]
\subsubsection{Перевод из 10-ой системы счисления в n-ую}\label{sec:base_10_to_base_n}
Такой перевод осуществляется по следующему алгоритму:
\begin{enumerate}
    \item Поделить число на основание нужной СС;
    \item Если частное не 0, повторяем шаг 1 для частного;
    \item Получившиеся остатки запишем в обратном порядке;
\end{enumerate}
Приведем пример перевода $222_{10}$ в 5-ную СС:
\begin{figure}[h!]
    \begin{minipage}{0.18\linewidth}
    $$\arraycolsep=0.01em
    \begin{array}{rrr@{\,}r|l}
    2&2&2&&\,5\\
    \cline{5-5}
    2&0&&&\,\textcolor{red}{44}\\
    \cline{1-2}
    &2&2&&\\
    &2&0&&\\
    \cline{2-3}
    &&\textcolor{brown}{2}&&\\
    \end{array}$$
    \end{minipage}
    \hspace{1cm}
    \begin{minipage}{0.18\linewidth}
    $$\arraycolsep=0.01em
    \begin{array}{rr@{\,}r|l}
    \textcolor{red}{4}&\textcolor{red}{4}&&\,5\\
    \cline{4-4}
    4&0&&\,\textcolor{blue}{8}\\
    \cline{1-2}
    &\textcolor{brown}{4}&&\\
    \end{array}$$
    \end{minipage}
    \hspace{1cm}
    \begin{minipage}{0.18\linewidth}
    $$\arraycolsep=0.01em
    \begin{array}{r@{\,}r|l}
    \textcolor{blue}{8}&&\,5\\
    \cline{3-3}
    5&&\,\textcolor{orange}{1}\\
    \cline{1-1}
    \textcolor{brown}{3}&&\\
    \end{array}$$
    \end{minipage}
    \hspace{1cm}
    \begin{minipage}{0.18\linewidth}
    $$\arraycolsep=0.01em
    \begin{array}{r@{\,}r|l}
    \textcolor{orange}{1}&&\,5\\
    \cline{3-3}
    0&&\,0\\
    \cline{1-1}
    \textcolor{brown}{1}&&\\
    \end{array}$$
    \end{minipage}
\end{figure}\\
Таким образом, $222_{10} = {\textcolor{brown}{1342}_5}$. Заметим, что результат такой же, как и в \autoref{sec:base_n_to_base_10}. \\ 
Перевод нецелого десятичного числа в другую СС осуществляется по следующему алгоритму:
\begin{enumerate}
    \item Переведем целую часть в новую СС;
    \item Нецелую часть умножаем на новое основание;
    \item Целую часть результата запоминаем;
    \item Повторяем шаг 2 и 3, если не достигнута точность или получился целый результат;
\end{enumerate}
Для примера, переведем $222,37_{10}$ в пятеричную СС: целая часть $222_{10} = 1342_5$, нецелую $0,37$ переведем согласно алгоритму (точность: 3 знака после запятой):
\[0,37 * 5 = \textcolor{red}{1}\textcolor{orange}{,85}; \,\, \textcolor{orange}{0,85} * 5 = \textcolor{blue}{4}\textcolor{brown}{,25}; \,\, \textcolor{brown}{0,25} * 5 = \textcolor{green}{1},25\]\\
Нецелая часть получается: 0,\textcolor{red}{1}\textcolor{blue}{4}\textcolor{green}{1}. Получаем: $222,37_{10} \approx 1342,141_5$. Используется знак \textit{приблизительно равно}, так как мы остановились на вычислении определенной точности, т.е. возможно получить еще более точное значение.
\subsubsection{Перевод из n-ой СС в m-ую СС}\label{sec:base_n_to_base_m}
Используя техники из \autoref{sec:base_n_to_base_10} и \autoref{sec:base_10_to_base_n} попробуем перевести число $2855_9$ в 16-ную СС:
\begin{enumerate}
    \item Переведем число $2851_9$ в 9-ную:
    \[2855_9 = 2 * 9^3 + 8 * 9^2 + 5 * 9^1 + 5 * 9^0 = 2 * 729 + 8 * 81 + 5 * 9 + 5 = 2156_{10}\]
    \item Теперь переведем получившееся число в 16-ную:
    \begin{figure}[h!]
    \begin{minipage}{0.18\linewidth}
    $$\arraycolsep=0.01em
    \begin{array}{rrrr@{\,}r|l}
    2&1&5&6&&\,16\\
    \cline{6-6}
    1&6&&&&\,\textcolor{red}{134}\\
    \cline{1-2}
    &5&5&&&\\
    &4&8&&&\\
    \cline{2-3}
    &&7&6&&\\
    &&6&4&&\\
    \cline{3-4}
    &&\textcolor{brown}{1}&\textcolor{brown}{2}&&\\
    \end{array}$$
    \end{minipage}
    \hspace{1cm}
    \begin{minipage}{0.18\linewidth}
    $$\arraycolsep=0.01em
    \begin{array}{rrr@{\,}r|l}
    \textcolor{red}{1}&\textcolor{red}{3}&\textcolor{red}{4}&&\,16\\
    \cline{5-5}
    1&2&8&&\,\textcolor{blue}{8}\\
    \cline{1-3}
    &&\textcolor{brown}{6}&&\\
    \end{array}$$
    \end{minipage}
    \hspace{1cm}
    \begin{minipage}{0.18\linewidth}
    $$\arraycolsep=0.01em
    \begin{array}{r@{\,}r|l}
    \textcolor{blue}{8}&&\,16\\
    \cline{3-3}
    0&&\,0\\
    \cline{1-1}
    \textcolor{brown}{8}&&\\
    \end{array}$$
    \end{minipage}
\end{figure}\\
\end{enumerate}
Запишем остатки в обратном порядке: $86C_{16}$ (Помним, что если остаток больше 10, то просто используем соответсвующую его значению цифру из \autoref{tab:latin_digits}). Таким образом, $2855_9 = 86C_{16}$.
\subsubsection{Связь 2-ой, 4-ой, 8-ой и 16-ой систем счислений}
(Мы знаем, что $2^1=2;\,\, 2^2=4;\,\, 2^3=8;\,\, 2^4=16$)\\
При необходимости перевода из 2-ной СС в 4-ую, 8-ую или 16-ую, можно воспользоваться следующим свойством этих СС: числа, образованные каждыми $m$ цифрами числа в бинарном представлении эквиваленты одной цифре в новой СС, $m$ - показатель степени с основанием 2 равный основанию СС, в которую переводим.\\
Для примера, переведем $101110100_2$ в 8-ую ($2^3 = 8$, поэтому выделяем по 3 цифры):
\[
  \underbrace{101}_5
  \!
  \underbrace{110}_6
  \!
  \underbrace{100}_4
\]
Следовательно, $101110100_2 = 564_8$. Числа образованные каждым блоком цифр переводим, как обычное число в 10-ую (\autoref{sec:base_n_to_base_10}). Если число нельзя поделить на блоки по 2, 3 или 4 (в зависимости от основания СС, в которую переводим), то припишем необходимое количество незначащих нулей в начало число и будем переводить, главное начать деление с правого конца. Например, переведем $11100_2$ в 16-чную:
\begin{enumerate}
    \item Добавим еще три нуля в начало, чтобы число делилось на блоки по 4 цифры: $00011100_2$;
    \item Разделим на блоки по 4 и переведем:
    \[
    \underbrace{0001}_1
    \!
    \underbrace{1100}_{12}
    \]
\end{enumerate}
Следовательно, $11100_2 = 1C_{12}$ (не забываем, что 12 в данном контексте цифра, поэтому заменяем согласно \autoref{tab:latin_digits}).\\
Этим же свойством можно воспользоваться при обратном переводе: необходимо каждую цифру исходного числа представить как $m$ цифр числа в нужной СС, $m$ - показатель степени с основанием 2 равный основанию СС, в которую переводим.  Для примера, переведем $3112_4$ ($2^2=4$, следовательно, каждая цифра будет записана как две) в бинарную СС:
\[3112_4 = \underbrace{11}_3 \! \underbrace{01}_1 \! \underbrace{01}_1 \! \underbrace{10}_2\]
Следовательно, $3112_4 = 11010110_2$. Обратите внимание, что в отдельных блоках могут возникать незначащие нули, однако убрать их нельзя, так как в общей записи они являются значащими (кроме нулей в первом блоке).\\
Данное свойство работает не только при переводе из/в 2-ую, но и между, например, 16-ной и 8-ой или 8-ой и 4-ой, в таких случаях блок или цифры представляются как цифры/блоки в соответствующих СС.
\subsubsection{Системы счисления в Python}
По умолчанию в Python все вычисления происходят в 10-ой СС, однако существует ряд встроенных функций, которые позволяют использовать иные СС.\\
Например, для перевода 10-ного числа в двоичную СС, можно воспользоваться функцией \href{https://docs.python.org/3/library/functions.html#bin}{\texttt{bin()}}:
\begin{minted}[linenos]{python}
a = 124
b = bin(a)
print(b)
\end{minted}
Результат:
\begin{minted}{python}
0b1111100
\end{minted}
Результат — строка с двоичным представлением числа.
Первые два символа \mintinline{python}{'0b'} нужны лишь для обозначения записи, поэтому можем от них избавиться применив срез:
\begin{minted}[linenos]{python}
a = 124
b = bin(a)[2:]
print(b)
\end{minted}
Результат:
\begin{minted}{python}
1111100
\end{minted}
Такие же функции предусмотрены и для 8-ной и 16-ой СС:
\begin{minted}[linenos]{python}
a = 239
b = oct(a)
c = hex(a)
print(b, c)
\end{minted}
Результат:
\begin{minted}{python}
0o357 0xef
\end{minted}
Префикс у восьмеричного представления - \mintinline{python}{'0o'}, а у шестнадцатеричного - \mintinline{python}{'0x'}. Их так же можно убрать используя срез.\\
Встроенных функций для перевода из 10-ной СС в другие СС (кроме оснований 2, 8, 16) не существует. Однако можно самостоятельно написать такую функцию:
\begin{minted}[linenos]{python}
def to_base(n, base):
    digits = "0123456789ABCDEF"
    r = ""
    while n > 0:
        r = digits[n % base] + r
        n //= base
    return r
\end{minted}
Данная функция принимает на вход два параметра — само число и основание новой СС. Возвращает строку, с записью числа в нужной СС. Пример:
\begin{minted}[linenos]{python}
def to_base(n, base):
    digits = "0123456789ABCDEF"
    r = ""
    while n > 0:
        r = digits[n % base] + r
        n //= base
    return r

a = to_base(4522, 5)
print(a)
\end{minted}
Результат:
\begin{minted}{python}
121042
\end{minted}
То есть, $4522_{10} = 121042_5$. Необходимо помнить, что все функции, переводящие в другие СС (встроенная и написанная самостоятельно) возвращают строки, следовательно, с ними нельзя производить арифметические операции.\\
Для перевода в СС с основанием 10, можем использовать встроенную функцию \href{https://docs.python.org/3/library/functions.html#int}{\texttt{int()}}:
\begin{minted}[linenos]{python}
a = "124"
b = int(a, 5)
print(b)
\end{minted}
Результат:
\begin{minted}{python}
39
\end{minted}
То есть $124_5 = 39_{10}$. Первым параметром является само число (как строка) в данной СС, а вторым параметром (как целое число) само основание исходной СС ($\in \langle 2; 36 \rangle $). Данная функция возвращает целое число, поэтому с ним можно производить арифметические операции.
\subsection{Операции в разных СС с одной переменной}\label{sec:one_var_many_base}
\begin{problem}
Операнды арифметического выражения записаны в системе счисления с основаниями 15 и 13:
\[4Cx4_{15} + x62A_{13}\]
В записи чисел переменной x обозначена неизвестная цифра из алфавита десятичной системы счисления. Определите наименьшее значение x, при котором значение данного арифметического выражения кратно 121. Для найденного значения x вычислите частное от деления значения арифметического выражения на 121 и укажите его в ответе в десятичной системе счисления. Основание системы счисления в ответе указывать не нужно.
\begin{solution}
Сначала решим без программирования. Чтобы не возникало путаницы, напомним себе, что запись $\overline{abc}$ обозначает число, где на соответствующих позициях стоят цифры a, b и c.\\ Любое число в позиционной СС можно записать с помощью \autoref{eq:pos_nums_1}:\\ Тогда, \(\overline{4Cx4_{15}} = 4 * 15^3 + 12 * 15^2 + x * 15 + 4 = 13500 + 2700 + 15x + 4 = 16204 + 15x\)\\ \(\overline{x62A_{13}} = x * 13^3 + 6 * 13^2 + 2 * 13 + 10 = 2197x + 1014 + 26 + 10 = 1050 + 2197x\)\\
Сложим две суммы и получим:\\
\(\overline{4Cx4_{15}} + \overline{x62A_{13}} = 16204 + 15x + 1050 + 2197x = 17254 + 2212x\)\\
Получившееся выражение записано в десятичной СС. Поставим ограничения на x:
\begin{itemize}
    \item По условию х из алфавита десятичной СС, следовательно $0 < x < 10$;
    \item Основание СС у второго слагаемого 13, следовательно $x < 13$;
    \item Основание СС у первого слагаемого 15, следовательно $x < 15$;
    \item Во втором числе, x стоит на первом месте, а значит $x \neq 0$;
\end{itemize}
Получаем, $x \in \langle 1; 9 \rangle $. Переберем значения в порядке возрастания (так как ищем наименьший возможный x):
\begin{enumerate}
    \item $x = 1;\, 17254 + 2212 = 19466 \centernot\vdots 121$
    \item $x = 2;\, 17254 + 4424 = 21678 \centernot\vdots 121$
    \item $x = 3;\, 17254 + 6636 = 23890 \centernot\vdots 121$
    \item $x = 4;\, 17254 + 8848 = 26102 \centernot\vdots 121$
    \item $x = 5;\, 17254 + 11060 = 28314 \vdots 121$
\end{enumerate}
Следовательно, $x = 5; 28314 \div 121 = 234$.\\
\textbf{Ответ:} 234\\
Однако данный способ требует большого количества ручных вычислений, поэтому занимает много времени. Попробуем записать программу для автоматического подбора подходящего x:
\begin{minted}[linenos]{python}
for x in '123456789': # перебираем все возможные значения x
    a = int('4C' + x + '4', 15) # составляем первое число и переводим в 10-ную СС
    b = int(x + '62A', 13) # составляем второе число и переводим в 10-ную СС
    t = a + b # вычисляем значение выражения
    if t % 121 == 0: # проверяем, делится ли на 121
        print(t // 121) # выводим результат при делении
        break # выходим из цикла
\end{minted}
Достаточно сказать о двух вещах:
\begin{enumerate}
    \item Перебираем x строго из диапазона возможных значений ($x \in \langle 1; 9 \rangle $);
    \item При первом найденном числе — выходим из цикла. Так как мы перебираем значения x в порядке возрастания, то первое подходящее — наименьшее значение x из возможных, а значит нет смысла искать следующее.
\end{enumerate}
\end{solution}
\end{problem}

\begin{problem}
Операнды арифметического выражения записаны в системе счисления с основаниями 16 и 14:
\[
3D4x_{16} + 4xC4_{14}
\]
В записи чисел переменной x обозначены допустимые в данных системах счисления неизвестные цифры. Определите наименьшее значение x, при котором значение данного арифметического выражения кратно 154. Для найденного значения x вычислите частное от деления значения арифметического выражения на 154 и укажите его в ответе в десятичной системе счисления. Основание системы счисления в ответе указывать не нужно.
\begin{solution}
Поставим ограничения на x:
\begin{enumerate}
    \item x присутствуют в записи числа в 16-ной СС, следовательно $0 \leqslant x < 16$
    \item x присутствуют в записи числа в 14-ной СС, следовательно $0 \leqslant x < 14$
\end{enumerate}
Следовательно, $x \in \langle 0; 13 \rangle$. Будем использовать шаблон:
\begin{minted}[linenos]{python}
for x in '0123456789ABCD': # перебираем все возможные значения x
    a = int('3D4' + x, 16) # составляем первое число и переводим в 10-ную СС
    b = int('4' + x + 'C4', 14) # составляем второе число и переводим в 10-ную СС
    t = a + b # вычисляем значение выражения
    if t % 154 == 0: # проверяем, делится ли на 154
        print(t // 154) # выводим результат при делении
        break # выходим из цикла
\end{minted}
Результат:
\begin{minted}{python}
187
\end{minted}
\textbf{Ответ:} 187
\end{solution}
При решении данного подтипа, используем следующий алгоритм действий:
\begin{enumerate}
    \item Установить ограничения на x в зависимости от условия задачи и операндов;
    \item Записать шаблон решения;
    \item Записать правильный диапазон для перебора значений x. Если ищем минимальный подходящий — выставляем в порядке возрастания, в обратном случае - в порядке убывания;
    \item Правильно записать составление каждого операнда и соответствующие им основания СС;
    \item Правильно записать проверку на делимость и удостовериться, в выводе нужного значения и выхода из цикла;
\end{enumerate}
\end{problem}
\subsection{Операции в разных СС с двумя переменными}\label{sec:two_var_many_base}
\begin{problem}
Операнды арифметического выражения записаны в системах счисления с основаниями 9 и 11:
\[
88x4y_9 + 7x44y_{11}
\]
В записи чисел переменными x и y обозначены допустимые в данных системах счисления неизвестные цифры. Определите значения x и y, при которых значение данного арифметического выражения будет наименьшим и кратно 61. Для найденных значений x и y вычислите частное от деления значения арифметического выражения на 61 и укажите его в ответе в десятичной системе счисления. Основание системы счисления в ответе указывать не нужно.
\begin{solution}
Данный подтип задачи похож на \autoref{sec:one_var_many_base}, однако теперь мы ищем минимальное значение выражения (не переменной), а так же количество переменных возрастает до двух.\\
Может показаться, что достаточно изменить шаблон, добавив вложенный цикл для подбора второй переменной, однако это не всегда верно. Так как позиция переменной в числе влияет на значение, то однозначно определить порядок перебора не всегда легко, рассмотрим следующий пример: $\overline{b3a}$ и $\overline{ab3}$. Очевидно, чем левее разряд, тем он весомее, однако переменные на самых левых позициях меняются, при этом позиции на следующих разрядах тоже не однозначные. Чтобы исключить возможность неправильного порядка перебора, будем его производить в любом порядке (главное перебрать все варианты) и каждый результат выражения записывать в список, интересующее нас значение — минимум в списке:
\begin{minted}[linenos]{python}
l = [] # создаем пустой список
for x in '012345678': # перебираем все возможные значения x
    for y in '012345678': # перебираем все возможные значения y
        a = int('88' + x + '4' + y, 9) # составляем первое число и переводим в 10-ную СС
        b = int('7' + x + '44' + y, 11) # составляем второе число и переводим в 10-ную СС
        t = a + b # вычисляем значение выражения
        if t % 61 == 0: # проверяем, делится ли на 61
            l.append(t) # добавляем результат выражения в список
print(min(l) // 61) # выводим частное от деления МИНимального значения на 61
# print(max(l) // 61) # если необходимо частное при делении МАКСимального значения на 61
\end{minted}
Результат:
\begin{minted}{python}
2715
\end{minted}
\textbf{Ответ:} 2715
\end{solution}
\end{problem}
Обратите внимание на следующие отличия от шаблона из \autoref{sec:one_var_many_base}:
\begin{enumerate}
    \item Перед циклом создается пустой список, туда будем добавлять значения выражений;
    \item Добавляется вложенный цикл для подбора значения второй переменной (вложенный цикл обеспечивает перебор всех возможных комбинаций);
    \item Если значение выражения делится на необходимое число, то оно просто добавляется в список: \textit{не выводится и цикл не прерывается (не пишем \mintinline{python}{break})};
    \item Находим минимум и вычисляем частное только после всех циклов (нет отступов);
\end{enumerate}
При решении необходимо верно определить ограничения для каждой переменной \textit{независимо от другой}. Когда определяем ограничения на x, представляем, что y просто какая-то цифра и не обращаем на нее внимание. То же самое повторяем для y.
\begin{problem}
Операнды арифметического выражения записаны в системах счисления с основаниями 8 и 11:
\[
y04x5_{11} + 253xy_8
\]
В записи чисел переменными x и y обозначены допустимые в данных системах счисления неизвестные цифры. Определите значения x и y, при которых значение данного арифметического выражения будет наименьшим и кратно 117. Для найденных значений x и y вычислите частное от деления значения арифметического выражения на 117 и укажите его в ответе в десятичной системе счисления. Основание системы счисления в ответе указывать не нужно.
\begin{solution}
Поставим ограничения на каждую из переменных по отдельности:
\begin{itemize}
    \item Переменная x есть в обоих числах, минимальное основание СС — восемь, следовательно: $0 \leqslant x < 8$;
    \item Переменная y есть в обоих числах, минимальное основание СС — восемь. В первом операнде стоит на первом месте, следовательно: $1 \leqslant x < 8$;
\end{itemize}
Запишем программу используя шаблон:
\begin{minted}[linenos]{python}
l = [] # создаем пустой список
for x in '01234567': # перебираем все возможные значения x
    for y in '1234567': # перебираем все возможные значения y
        a = int(y + '04' + x + '5', 11) # составляем первое число и переводим в 10-ную СС
        b = int('253' + x + y, 8) # составляем второе число и переводим в 10-ную СС
        t = a + b # вычисляем значение выражения
        if t % 117 == 0: # проверяем, делится ли на 117
            l.append(t) # добавляем результат выражения в список
print(min(l) // 117) # выводим частное от деления минимального значения на 117
\end{minted}
Результат:
\begin{minted}{python}
224
\end{minted}
\textbf{Ответ:} 224
\end{solution}
\end{problem}
Алгоритм решения данного подтипа следующий:
\begin{enumerate}
    \item Найти ограничения на каждую из переменных по отдельности;
    \item Записать шаблон решения;
    \item В цикле для каждой переменной правильно записать ее диапазон значений;
    \item Записать правильное условие (делимость на необходимое число);
    \item Проверить, что добавляем \textit{результат выражения} в список и \textit{НЕ завершаем} цикл;
    \item Проверить, ищем ли минимальное или максимальное значение и \textit{вывести частное};
\end{enumerate}
\subsection{Операции в одной СС}
Задачи из данной секции обобщают идеи из \autoref{sec:one_var_many_base} и \autoref{sec:two_var_many_base}, однако теперь вычисления происходят в одной конкретной СС, при этом количество переменных может меняться.\\
Не для всех задач данного раздела существует четкий шаблон. Однако самостоятельное решение в большинстве задач просто объединяет приемы из шаблонов и общих идей в программировании.
\begin{problem}
Операнды арифметического выражения записаны в системе счисления с основанием 15:
\[123x5_{15} + 1x233_{15}\]
В записи чисел переменной x обозначена неизвестная цифра из алфавита 15-ричной системы счисления. Определите наименьшее значение x, при котором значение данного арифметического выражения кратно 14. Для найденного значения x вычислите частное от деления значения арифметического выражения на 14 и укажите его в ответе в десятичной системе счисления. Основание системы счисления в ответе указывать не нужно.
\begin{solution}
Подобные этой задаче решаются с шаблоном из \autoref{sec:one_var_many_base}:
\begin{minted}[linenos]{python}
for x in '0123456789ABCDE': # перебираем все возможные значения x
    a = int('123' + x + '5', 15) # составляем первое число и переводим в 10-ную СС
    b = int('1' + x + '233', 15) # составляем второе число и переводим в 10-ную СС
    t = a + b # вычисляем значение выражения
    if t % 14 == 0: # проверяем, делится ли на 14
        print(t // 14) # выводим результат при делении
        break # выходим из цикла
\end{minted}
Результат:
\begin{minted}{python}
8767
\end{minted}
Важно обращать внимание на ограничения каждой переменной (в данном случае только одна - x):
\begin{itemize}
    \item По условию, x из алфавита 15-ричной СС (цифры в 15-ричной СС), следовательно $x \in \langle 0; 14 \rangle$. Эквиваленты цифровым значениям найдем \autoref{tab:latin_digits};
    \item Перебираем значения x в порядке возрастания, выходим из цикла как только выражение подходит;
    \item Оба числа переводятся из 15-ричной СС;
\end{itemize}
\textbf{Ответ:} 8767
\end{solution}
\end{problem}


\begin{problem}
Числа M и N записаны в системе счисления с основанием 9 соответственно.
\[M = 842x5_9, \,\, N = 8x725_9\]
В записи чисел переменной x обозначена неизвестная цифра из алфавита девятеричной системы счисления. Определите наименьшее значение натурального числа A, при котором существует такой x, что M + A кратно N.
\begin{solution}
Подобрать сразу несколько значений можно с помощью вложенного цикла, однако стоит верно определить порядок перебора. Посмотрим на конструкцию:
\begin{minted}{python}
for A in range(5):
    for x in range(3):
        print(A, x)
\end{minted}
Полученные значения:\\ \mintinline{python}{(0, 0); (0, 1); (0, 2); (1, 0); (1, 1); (1, 2); ... (4, 0); (4, 1); (4, 2)}\\
Можно сказать, что для каждого значения A перебираются все значения x: \[(A_0, x_0), (A_0, x_1), ... (A_0, x_n); (A_1, x_0), (A_1, x_1), ... (A_1, x_n);\,...\,; (A_m, x_0), (A_m, x_1), ... (A_m, x_n);\]
Если поменять местами циклы, получим, что для каждого значения x перебираются все значения A.\\
Так как в первую очередь нас интересует минимальный A, будем перебирать все значения x для каждого A в порядке возрастания. Как только какая-то пара (A, x) образует равенство $(M + A)\,mod\,N = 0$, завершим программу:
\begin{minted}[linenos]{python}
for A in range(1, 10000): # Перебираем возможные значения A
    for x in '012345678': # Перебираем все возможные x
        M = int('842' + x + '5', 9) # Составляем M
        N = int('8' + x + '725', 9) # Составляем N
        if (M + A) % N == 0: # Если (M + A) делится на N
            print(A) # Выводим А
            exit(0) # Завершаем программу
\end{minted}
Следует обратить внимание на следующие пункты:
\begin{itemize}
    \item $A \in \mathbb{N}$, следовательно, начинаем перебор с 1, не с 0 ($\mathbb{N}=\{1,2,3,\ldots,\infty\}$). Верхняя граница не может быть больше, чем максимум из двух чисел (почему?), необходимо просто взять достаточно большое число;
    \item Правильно определяем ограничения на x, с учетом основания СС и позиций x. В данном случае $x \in \langle 0; 8 \rangle$;
    \item Используем \mintinline{python}{exit(0)} вместо \mintinline{python}{break}, так как завершаем сразу всю программу, а не только внутренний цикл (почему?);
\end{itemize}
\end{solution}
\end{problem}


\begin{problem}
В числе $58x723y49_{39}$ x и y обозначают некоторые цифры из алфавита системы счисления с основанием 39. Определите такие значения x и y, при которых приведённое число кратно 38, а число $yx_{39}$ является максимально возможным полным квадратом. В ответе запишите значение числа $yx_{39}$ в десятичной системе счисления.
\begin{solution}
Основная проблема, которая может возникнуть при решении данной задачи заключается в том, что встроенная функция \href{https://docs.python.org/3/library/functions.html#int}{int()} не поддерживает СС, чьи основания превосходят 36 - наш случай.\\
В данной задаче необходимо использовать перевод из n-ой CC в 10-чную СС используя \autoref{eq:pos_nums_1}:
\begin{enumerate}
    \item Пронумеруем разряды: $\underset{8}{5} \underset{7}{8} \underset{6}{x} \underset{5}{7} \underset{4}{2} \underset{3}{3} \underset{2}{y} \underset{1}{4} \underset{0}{9}$;
    \item Будем использовать формулу:
    \[5 * 39^8 + 8 * 39^7 + x * 39^6 + 7 * 39^5 + 2 * 39^4 + 3 * 39^3 + y * 39^2 + 4 * 39^1 + 9 * 39^0\]
    \item Число $yx_{39}$ будем переводить в 10-чную СС используя формулу: $y * 39^1 + x * 39^0$;
\end{enumerate}
Запишем программу с перебором x и y:
\begin{minted}[linenos, breaklines]{python}
import math # подключаем библиотеку с мат. функциями

l = [] # создаем пустой список
for x in range(0, 39): # перебираем x
    for y in range(0, 39): # перебираем y
        n = 5 * 39**8 + 8 * 39**7 + x * 39**6 + 7 * 39**5 + 2 * 39**4 + 3 * 39**3 + y * 39**2 + 4 * 39 + 9 # исходное число в 10-чную
        m = y * 39 + x # второе число в 10-чную
        if math.sqrt(m)**2 == m and (n % 38) == 0: # проверка, что m полный квадрат и исходное кратно 38
            l.append(m) # добавляем число в список
print(max(l)) # выводим максимум из всех
\end{minted}
Обратите внимание на следующие пункты:
\begin{itemize}
    \item x и y перебираются как числа, не в виде строки (\mintinline{python}{'0123...'}, но в виде числового эквивалента, поэтому используем \mintinline{python}{range()};
    \item Используем метод \href{https://docs.python.org/3/library/math.html#math.sqrt}{sqrt()} из встроенной библиотеки \href{https://docs.python.org/3/library/math.html}{math} для извлечения корня. Пользуемся свойством, что $a$ — полный квадрат, если $(\sqrt{a})^2 = a\,\,(a \geqslant 0)$;
    \item Не выходим из цикла при нахождении подходящих значений;
\end{itemize}
\end{solution}
\end{problem}
В данных задачах необходимо размышлять над тем, как 
% THE DOCUMENT IS ESSENTIALLY DONE AT THIS POINT, NO NEED TO EDIT ANYTHING BELOW THIS______________________________________________________________________________________________
\end{document}